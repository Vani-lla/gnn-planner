\Chapter{Algorytm}\label{chapter:algorytm}
\section{Wybór metod i technologi}
    Dlaczego takie i takie i takie metody. Dlaczego GoogleORTools jako solver liniowy. Wcześniejsze podejścia i dlaczego są gorsze.

\section{Struktura algorytmu}
    Krótki opis i diagram(y) na temat etapów algorytmu

\section{Algorytm zachłanny}
    \subsection{Bloki lekcyjne}
        Pierwszym krokiem w moim rozwiązaniu jest tworzenie bloków tak jak panu opowiadałem. Przykładowo Franc+Niem+Rus
    \subsection{Wpływ na złożoność problemu}
    \subsection{Działanie}

\section{Algorytm ewolucyjny}
    \subsection{Cel algorytmu}
        Drugim krokiem jest przydział każdego bloku do odpowiedniego dnia.
        W ten sposób rozbijam jeden wielki problem na 5 mniejszych problemów (jeden dla każdego dnia) którymi karmię solver liniowy.
    \subsection{Kodowanie}
    \subsection{Generowanie populacji}
    \subsection{Przystosowanie}
    \subsection{Krzyżowanie}
    \subsection{Mutacja}
    \subsection{Jaki jest oczekiwany najlepszy osobnik}

\section{Solver liniowy do układania planu dla każdego dnia osobno}
    \subsection{Czym są interwały}
        Wyjaśnienie jak reprezentuję bloki zajęć przez intervały
    \subsection{Ograniczenia}
        \begin{itemize}
            \item Nienakładanie się lekcji dla każdej klasy
            \item Nienakładanie się lekcji dla każdego nauczyciela
            \item Nienakładanie się lekcji w jendym pokoju
            \item Lekcje kończące/zaczynające
            \item Brak okienek dla uczniów
        \end{itemize}
    \subsection{Funkcja celu}
        Jak reprezentujemy okienka nauczycieli. Dlaczego interwały w tym pomagają.


\section{Wyniki}
    \begin{itemize}
        \item Dlaczego moje wyniki są wspaniałe
        \item Średni czas potrzebny na generację planu
    \end{itemize}

    \subsection{Statystyki planu}
        \begin{itemize}
            \item Ilość okienek
            \item Rozkład lekcji w tygodniu
            \item Lekcje początkujące/kończące
            \item Statystyki nauczycieli, godziny w szkole do godzin lekcyjnych (płatnych)
        \end{itemize}

    \subsection{Porównanie z ręcznie ułożonym planem}
        Porównanie z planem, który szkoła ułożyła ręcznie.