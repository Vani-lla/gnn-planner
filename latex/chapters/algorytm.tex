\Chapter{Problem układania planu lekcji i algorytm jego rozwiązania}\label{chapter:algorytm}
    Głównym elementem systemu jest wieloetapowy algorytm generowania planu lekcji.
    Jego główna innowacja leży w dekompozycji oryginalnego zadania na trzy sekwencyjne fazy.
    Zadaniem pierwszych dwóch jest zmniejszenie przestrzeni decyzyjnej do coraz mniejszej skali, tak aby w ostatniej fazie można było sformułować i rozwiązać problem programowania całkowitoliczbowego (MIP).
    Rezultatem takiego podejścia jest redukcja liczby zmiennych decyzyjnych, potrzebnych do zdefiniowania ograniczeń w trzecim etapie.

    Chociaż czyste podejście MIP prowadzi do teoretycznie optymalnych rozwiązań, w praktyce jego zastosowanie do pełnego problemu jest niemożliwe.
    Złożoność obliczeniowa i pamięciowa przekracza możliwości przeciętnych komputerów, co uniemożliwia efektywny rozwój takiego rozwiązania.
    Ponadto takie rozwiązanie jest też trudne do zaprogramowania ze względu na bloki lekcyjne.

    Zaprezentowane podejście hybrydowe pozwala pokonać problem złożoności obliczeniowej, oferując praktyczny kompromis między optymalnością a czasem obliczeń.

    \section{Sformułowanie problemu optymalizacyjnego}
        Na potrzeby pracy warto ujednolicić terminologię, z uwagi na to, że w języku potocznym niektóre z tych terminów są używane zamiennie:
        \begin{itemize}
            \item \textbf{Klasa}: Grupa uczniów; przykładowo ``IIA'', ``IVC'', \dots
            Ze względu na angielską nazwę \textit{class}, która koliduje ze składnią języków programowania, w kodzie często odnoszę się do klas jako \verb|student_group|.
            \item \textbf{Sala}: Miejsce, w którym prowadzone są zajęcia; przykładowo ``Sala Gimnastyczna 1'', ``2'', \dots
            \item \textbf{Przedmiot}: Temat zajęć prowadzonych przez nauczyciela; przykładowo ``Wychowanie Fizyczne'', ``Matematyka'', \dots
            \item \textbf{Lekcja}: Zajęcia prowadzone przez jednego nauczyciela, w jednej sali, z jedną lub więcej klas, które są na temat jednego przedmiotu.
            \item \textbf{Blok lekcyjny}: Grupa dwóch lub więcej lekcji, które są w tym samym okienku czasowym. 
            Mogą one dotyczyć jednej klasy oraz wielu nauczycieli, jednego nauczyciela i wielu klas, lub też wielu klas i wielu nauczycieli.
            \item \textbf{Okienko}: Przerwa między dwoma lekcjami klasy lub nauczyciela. Występuje gdy zajęcia nie są przeprowadzane bezpośrednio po sobie.
        \end{itemize}

        Problem optymalizacyjny w tej pracy polega na przypisaniu lekcji do odpowiednich slotów czasowych i sal przy jednoczesnym spełnieniu wymagań.
        W rzeczywistości sformułowanie takiego zadania i wyznaczenie jego rozwiązania stanowi duże wyzwanie.
        Istnieją ograniczenia, które są różne dla każdej klasy, co utrudnia formułowanie problemu --- wiele lekcji jest realizowanych w blokach, które są definiowane każdy z osobna.
        Przez te wyjątki nie jest możliwym wykorzystanie prostych algorytmów.
        Nie jest także możliwym rozwiązanie jednego wielkiego problemu programowania całkowitoliczbowego w sensownym czasie przy użyciu komputera z przeciętną specyfikacją.

        \begin{samepage}
            Oczekiwanym rezultatem działania algorytmu powinna być lista przypisań. Każde takie przypisanie powinno mieć 5 wartości:
            \begin{enumerate}
                \item Slot czasowy
                \item Klasa
                \item Sala
                \item Nauczyciel
                \item Przedmiot
            \end{enumerate}
            W przypadku lekcji, która obejmują więcej klas niż jedna, należy stworzyć przypisanie dla każdej klasy osobno.
        \end{samepage}

        \subsection{Ograniczenia}
            Wcześniej wspomniane ograniczenia można podzielić na 4 kategorie:

            \subsubsection{Ograniczenia fizyczne}
                \begin{itemize}
                    \item Żaden nauczyciel nie może być w 2 miejscach na raz.
                    \item Nauczyciel musi być dostępny.
                    Ze względu na charakter pracy nauczyciele często są zmuszeni do pracy w wielu miastach w wielu szkołach.
                    Układając plan musimy brać pod uwagę ich dostępność.
                    \item Żaden uczeń nie może być w 2 miejscach na raz.
                    \item W żadnej sali nie mogą odbywać się 2 lekcje na raz.
                \end{itemize}
            
            \subsubsection{Ograniczenia prawne}
                \begin{itemize}
                    \item Maksymalnie 9 godzin lekcyjnych dziennie.
                    \item Nie więcej niż 2 godziny lekcyjne tego samego przedmiotu dziennie.
                    \item Jeśli danego dnia mają zostać przeprowadzone 2 godziny jednego przedmiotu, to muszą one wystąpić bezpośrednio po sobie.
                \end{itemize}

            \subsubsection{Ograniczenia jakościowe}
                \begin{itemize}
                    \item Brak okienek dla uczniów.
                    \item Równomierny rozkład godzin na przestrzeni tygodnia. Nie może wystąpić sytuacja gdzie jednego dnia uczeń ma dwie lekcje, a następnego dziesięć.
                    \item Odpowiednie przypisanie sal. Lekcje wychowania fizycznego muszą odbyć się w przeznaczonych do tego salach, podobnie lekcji informatyki itd.
                \end{itemize}
            
            \subsubsection{Ograniczenie główne}\label{subsubsection:ograniczenie_glowne}
                Ilość godzin tygodniowo odbytych przez klasę z danym nauczycielem w ramach danego przedmiotu musi być równa ilości przypisanej w pierwszym etapie układania planu.
                Aby łatwiej zrozumieć na czym polega takie przypisanie warto spojrzeć na dotychczasowy sposób przypisywania ilości godzin nauczycieli do klas (Rysunek~\ref{fig:excel_wymagania}) w liceum, które dostarczyło dane na potrzebu tej pracy.
            
                \begin{figure}[ht]
                    \centering
                    \includegraphics[width=0.9\textwidth]{images/excel_wymagania.png}
                    \caption{Zrzut ekranu z arkusza kalkulacyjnego przedstawiąjący przypisanie godzinowe nauczycieli dla każdej klasy}\label{fig:excel_wymagania}
                \end{figure}
                Każdy nauczyciel jest przypisany do prowadzonych przez niego przedmiotów. 
                Następnie w odpowiednim wierszu nauczyciela, pod odpowiednim przedmiotem, w kolumnie każdej klasy definiowana jest ilość godzin, która będzie poświęcona na prowadzenie tego przedmiotu.

\section{Poprzednie podejścia}
    Aby w pełni zrozumieć mój wybór narzędzi warto szybko przetoczyć historię moich poprzednich podejść do rozwiązania tego problemu.
    
    \subsection{Programowanie zero-jedynkowe}
        Moim pierwszym podejściem była próba użycia tylko i wyłącznie solvera liniowego.
        Użyłem w tym celu pakietu \textit{IBM ILOG CPLEX}.
        Problem zdefiniowałem używając zmiennych binarnych, tworząc sześcio wymiarową macierz:
        \begin{enumerate}
            \item Wymiar nauczycieli
            \item Wymiar klas
            \item Wymiar przedmiotów
            \item Wymiar dnia
            \item Wymiar slotu czasowego
            \item Wymiar sal
        \end{enumerate}

        Jak można zauważyć złożoność pamięciowa takiego podejścia uniemożliwia jego efektywne skalowanie.
        Nawet dla danych średniej szkoły, mającej mniej niż 100 sal, nauczycieli, klas i przedmitów, taka macierz zajmowała setki GB pamięci RAM.
        Rozwiązaniem tego problemu było zastosowanie słownika z wartościami jako zmienne binarne i kluczami jako krotki 6 liczb całkowitych.
        W ten sposób pozbywam się wszystkich niemożliwych wartości, przykładowo wychowania fizycznego z nauczycielem matematyki.
        Używając tej metody nadal możemy używać intuicji, która towarzyszy z macierzą.
        Musimy tylko sprawdzać czy dane zmienne binarne faktycznie istnieją.

        Jest to intuicyjny sposób poradzenia sobie z problemem harmonogramowania zajęć o stałem długości. 
        Bardzo łatwo można definiować ograniczenia fizyczne. Przykładowo ograniczenie prowadzenia maksymalnie jednej lekcji dla wszystkich nauczycieli:
        \[ \forall t \in \left\{0, 1, 2, \dots, t_\text{max}\right\} \quad \sum_{c=0}^{c_\text{max}} \sum_{s=0}^{s_\text{max}} \sum_{d=0}^{4} \sum_{h=0}^{h_\text{max}} \sum_{r=0}^{r_\text{max}} x_{t,c,s,d,h,r} = 1 \]
        Gdzie:
        \begin{itemize}
            \item $t$ to indeks nauczyciela i $t_\text{max}$ to liczba nauczycieli minus jeden (indeksowanie zaczynamy od 0)
            \item $c$ to indeks klasy i podobnie $c_\text{max}$ to liczba klas minus jeden
            \item $d$ to dzień, gdzie 1 oznacza poniedziałek, a 4 piątek
            \item $h$ to slot czasowy, a $h_\text{max}$ to horyzont
            \item $r$ to indeks sali, a $r_\text{max}$ to liczba wszystkich sal minus jeden
        \end{itemize}

        Problemem tego rozwiązania jest niemożność wprowadzenia bloków lekcyjnych przy jednoczesnym zachowaniu prostoty obliczeniowej.
        Bardzo ciężkim okazało się również wprowadzenie minimalizacji liczby okienek.
        Zmienne binarne nie oferują wystarczającej wszechstronności.
    
    \subsection{Grafowe sieci neuronowe}
        W odpowiedzi na problemy z MIP, zdecydowałem się na zbadanie alternatywnych metod rozwiązania.
        Wybrałem niekonwencjonalną reprezentację problemu harmonogramowania używając grafowych sieci neuronowych~\cite{schlichtkrull2018modeling}.
        W przyjętym modelu problem został przedstawiony w postaci grafu, gdzie węzły reprezentowały wymagania główne (\ref{subsubsection:ograniczenie_glowne}), 
            a krawędzie łączyły zajęcia o tych samych nauczycielach lub tych samych klasach.
        
        Taka reprezentacja okazała się szczególnie atrakcyjna pod względem implementacji funkcji celu.
        Ocena dopuszczalności rozwiązania sprowadzała się do weryfikacji spełnienia ograniczeń, które można było w prosty sposób zamodelować za pomocą funkcji kary
        Jednocześnie można nagradzać model za przypisywanie lekcji do poprawnych bloków.
        Początkowe rezultaty były bardzo obiecujące --- model już po kilkadziesięciu epokach wykazywał zdolność do identyfikowania, które lekcji powinny być w blokach, a które wymagają rozdzielenia w celu uniknięcia kolizji.

        Główną wadą w tym podejściu okazał się brak gwarancji spełnienia ograniczeń twardych oraz trudności przy układaniu planu iteracyjnie (lekcja po lekcji).
        Eksperymenty wykazały, że przy zastosowaniu zbyt wysokich współczynników kary, proces uczenia nie przynosił rezultatów. 
        Zbyt niskie natomiast powodowały, że model preferował optymalizację nagrody za grupowanie lekcji kosztem naruszenia ograniczeń.

\section{Wybór metod i technologi}
    W świetle przeprowadzonych eksperymentów i poprzednich prób rozwiazania problemu zauważyłem, że poleganie wyłącznie na metodach inteligentnych lub MIP nie doprowadzi do sensownych rezultatów.
    Zdecydowałem się na użycie:
    \begin{itemize}
        \item Algorytmu zachłannego do łączenia lekcji w bloki, gdyż jest to najefektywniejsza i najprostsza metoda do tego zadania.
        \item Algorytmu ewolucyjnego do przydziału godzin do każdego dnia tygodnia.
        \item Solvera liniwego do ułożenia samego planu dla każdego dnia tygodnia osobno.
    \end{itemize}
    Ze względu na prostotę integracji z backendem aplikacji użyłem języka programowania Python.
    Otwarta natura narzędzia Google OR-Tools oraz jego wygodne API w Pythonie skłoniło mnie do decyzji przeciwko CPLEX i Gurobi.

\section{Dane i parametry}
    Opis działania algorytmu warto zacząć od przedstawienia sposobu reprezentacji wymagań w kodzie zaczynając od wymagań głównych.
    Zaczynając procedurę generowania planu lekcji zaczynamy od pobrania z bazy danych odpowiednich rekordów.
    Do otrzymanych w ten sposób rezultatów należy: lista wymagań głównych, lista nauczycieli, lista sal, lista klas, lista przedmiotów, lista dostępności nauczycieli, 

    Podczas wszystkich etapów tworzenia planu lekcji mamy dostęp do list:
    \begin{itemize}
        \item ID nauczycieli,
        \item ID klas,
        \item ID przedmiotów,
        \item ID sal.
    \end{itemize}
    Ponadto mamy do dyspozycji listę wymagań głównych, gdzie każde z nich jest reprezentowane klasą~\cite{djangomodeldocs} w Pythonie, która ma poniższe atrybuty:
    \begin{itemize}
        \item ID nauczyciela,
        \item ID klasy,
        \item ID przedmiotu,
        \item niezerowa liczba godzin.
    \end{itemize}
    Każdy nauczyciel ma też zdefiniowaną dostępność, która jest reprezentowana przez macierz o wymiarach $T$ na $5$, gdzie $T$ to liczba wszystkich nauczycieli.
    Macierz jest zero-jedynkowa, gdzie 1 oznacza, że $t$-ty nauczyciel jest dostępny $i$-tego dnia tygodnia, a 0 że nie jest dostepny.
    \[ \begin{bmatrix}
        x_{1,1} & x_{1,2} & x_{1,3} & x_{1,4} & x_{1,5} \\
        x_{2,1} & x_{2,2} & x_{2,3} & x_{2,4} & x_{2,5} \\
        \vdots & \vdots & \vdots & \vdots \\  \\
        x_{T,1} & x_{T,2} & x_{T,3} & x_{T,4} & x_{T,5} \\
    \end{bmatrix}, \quad \forall t \in \left\{1, 2, \dots, T\right\}, \forall i \in \left\{1, 2, 3, 4, 5\right\}, \quad x_{t, i} \in \left\{0, 1\right\} \]
    W celu tworzenia bloków mamy także dostęp do klasy bloku przedmiotów, która ma następujące atrybuty:
    \begin{itemize}
        \item Listę ID przedmiotów.
        \item Listę liczby tych przedmiotów w bloku.
        \item Klas których dotyczy blok.
        \item Wartość binarną informującą o tym czy blok jest \textit{agregujący}.
        \item Liczbę naturalną z informacją o maksymalnej ilości takich bloków tygodniowo.
    \end{itemize}

\section{Struktura algorytmu}
    Pierwszy etap algorytmu służy do stworzenia bloków lekcyjnych.

\section{Algorytm zachłanny}
    \textit{Opowiedzenie czym jest algorytm zachłanny najlepiej w oparciu o jakąś pracę naukową.}

    \subsection{Bloki lekcyjne}
        Operując na blokach lekcyjnych duży łatwiej zdefiniować ograniczenia fizyczne.
        Weźmy na przykład 3 lekcje: Język Niemiecki, Język Francuski oraz Język Rosyjski.
        Gdybyśmy mieli definiować dla nich ograniczenie określające, że żaden uczeń nie może mieć przypisanych dwóch lub więcej lekcji w tym samym czasie, musielibyśmy zapewnić, że żadna z tych lekcji nie pokrywa się z innymi lekcjami tej klasy, takimi jak Matematyka czy Fizyka, 
            przy jednoczesnym zapewnieniu, że te lekcji mogą się na siebie nałożyć.
        Te zajęcia stanowią obowiązkowy język dodatkowy, który uczniowie wybierają każdy z osobna.
        Każdy uczeń może wybrać tylko jeden język, co za tym idzie lekcje mogą odbywać się niezależnie od siebie.
        Grupując takie ograniczenia główne w bloki 3 lekcji możemy potraktować taki blok jak każdą inną lekcję przy definiowaniu ograniczeń ---
        nie może być przypisany do tego samego slotu czasowego z żadną inną lekcją.

        Następną zaletą jest prostota w projektowaniu ograniczeń dla lekcji, które odbywają się dla więcej niż jednej klasy.
        Często w szkołach brakuje uczniów zapisanych na przykładowo Język Rosyjski w jednej klasie, aby uzasadnić indywidualną lekcję prowadzoną przez nauczyciela z tylko i wyłącznie jedną klasą.
        W takich przypadkach szkoła definiuje lekcje, które nauczyciel prowadzi dla wielu klas jednocześnie.
        Podobnie jak w poprzednim przykładzie, definiowanie ograniczeń dla każdej lekcji z osobna wiąże się z wyjątkami.
        Jeśli natomiast połączymy wymagania główne dla paru klas w jeden blok, możemy go traktować tak jak każdą inną lekcję.

        Kolejną zaletą tego rozwiązania jest możliwość łączenia bloków w jeszcze większe bloki.
        Wyobraźmy sobie sytuację, w której mamy trzy klasy: IIIA, IIIB i IIIC, oraz 3 przedmioty do przeprowadzenia: Język Niemiecki, Język Francuski i Język Rosyjski.
        Z uwagi na niską liczbę uczniów zapisanych na rosyjski i francuski te zajęcia sa prowadzone w następujących grupach:
        \begin{itemize}
            \item wszystkie 3 klasy mają razem Język Rosyjski,
            \item klasa IIIA i IIIB mają razem Język Francuski, a klasa IIIC ma indywidualnie z innym nauczycielem,
            \item każda klasa ma indywidualnie Język Niemiecki, z czego klasa IIIA i IIIC mają tego samego nauczyciela.
        \end{itemize}
        Jak można łatwo zauważyć wszystkie te lekcji poza jedną lekcją języka niemieckiego mogą odbyć się jednocześnie.
        Po stworzeniu wieloklasowych bloków lekcyjnych możemy je łączyć dalej.
        Język Rosyjski może być połączony z blokiem Języka Francuskiego dla klas IIIA i IIIB oraz lekcją klasy IIIC.
        Do tego możemy też dodać dwie lekcje języka Niemieckiego pozostawiając ostatnią lekcję poza blokiem ze względu na kolizję nauczycieli.

    \subsection{Działanie}
        \textit{Pseudokod i dokładne działanie.}

        Ważnym jest wyjście tego algorytmu:
        \begin{itemize}
            \item Lista bloków,
            \item Wymagana tygodniowa liczba godzin każdego bloku.
        \end{itemize}

        Efektem tych działań jest lista krotek takich wymagań oraz lista przypisań godzinowych.
        \begin{itemize}
            \item Lista bloków: $\left[\left(\text{Req 1}, \text{Req 2}, \text{Req 3}\right)\right]$
            \item Macierz wymaganych odbytych godzin każdego bloku na przestrzeni tygodnia. 
            \[V = \begin{bmatrix} v_1 & v_2 & \dots & v_B \end{bmatrix}, \quad \forall i \in \left\{1, 2, \dots, B\right\}, \quad v_i \in \mathbb{N} \]
            gdzie $B$ to liczba bloków.
        \end{itemize}
        \[  \]

\section{Algorytm ewolucyjny}
    \textit{Opowiedzenie czym jest algorytm ewolucyjny najlepiej w oparciu o jakąś pracę naukową.}

    \subsection{Cel algorytmu}
        Celem algorytmu jest przypisanie godzin lekcyjnych z macierzy $V$ do wszystkich roboczych dni tygodnia.
        Polega to na stworzeniu 5 wartości całkowitoliczbowych, jednej dla każdego dnia, reprezentujących liczbę godzin, przez które będzie odbywał się dany blok lekcyjny.

    \subsection{Kodowanie}
        Każdy osobnik jest reprezentowany przez macierz przypisań $P$ o wymiarach $5$ na $B$, gdzie $B$ reprezentuje liczbę wszystkich bloków.
        \[ P = \begin{bmatrix}
            p_{1,1} & p_{1,2} & p_{1,3} & \cdots & p_{1, B} \\
            p_{2,1} & p_{2,2} & p_{2,3} & \cdots & p_{2, B} \\
            p_{3,1} & p_{3,2} & p_{3,3} & \cdots & p_{3, B} \\
            p_{4,1} & p_{4,2} & p_{4,3} & \cdots & p_{4, B} \\
            p_{5,1} & p_{5,2} & p_{5,3} & \cdots & p_{5, B} \\
        \end{bmatrix}\]
        Taka reprezentacja pozwala na wygodne weryfikowanie ograniczeń:
        \begin{enumerate}
            \item Suma przypisań dla każdego dnia tygodnia musi być równa wymaganiom powstałych w wyniku działania poprzedniego algorytmu.
            \[ \forall i \in \left\{1, 2, \dots, B\right\}, \quad \sum_{d=1}^{5} p_{d,i} = v_d\]
            \item Maksymalna liczba poszczególnych bloków każdego dnia nie może być większa niż 2.
            \[ \forall i \in \left\{1, 2, \dots, B\right\}, \forall \quad \sum_{d=1}^{5} p_{d,i} = v_d\]
            \item Jeśli mamy do rozdzielenia 3 godziny danego bloku, to musiby je przydzielić do maksymalnie 2 dni.
            \item Uczniowie i nauczyciele muszą mieć równomierny rozkład bloków na przestrzeni tygodnia.
        \end{enumerate}
        
    \subsection{Generowanie populacji}
    \subsection{Przystosowanie}
    \subsection{Krzyżowanie}
    \subsection{Mutacja}
    \subsection{Jaki jest oczekiwany najlepszy osobnik}

\section{Solver liniowy do układania planu dla każdego dnia osobno}
    \subsection{Czym są interwały}
        Wyjaśnienie jak reprezentuję bloki zajęć przez intervały
    \subsection{Ograniczenia}
        \begin{itemize}
            \item Nienakładanie się lekcji dla każdej klasy
            \item Nienakładanie się lekcji dla każdego nauczyciela
            \item Nienakładanie się lekcji w jendym pokoju
            \item Lekcje kończące/zaczynające
            \item Brak okienek dla uczniów
        \end{itemize}
    \subsection{Funkcja celu}
        Jak reprezentujemy okienka nauczycieli. Dlaczego interwały w tym pomagają.


\section{Wyniki}
    \begin{itemize}
        \item Dlaczego moje wyniki są wspaniałe
        \item Średni czas potrzebny na generację planu
    \end{itemize}

    \subsection{Statystyki planu}
        \begin{itemize}
            \item Ilość okienek
            \item Rozkład lekcji w tygodniu
            \item Lekcje początkujące/kończące
            \item Statystyki nauczycieli, godziny w szkole do godzin lekcyjnych (płatnych)
        \end{itemize}

    \subsection{Porównanie z ręcznie ułożonym planem}
        Porównanie z planem, który szkoła ułożyła ręcznie.