\Chapter{Wstęp}\label{chapter:introduction}
    Jednym z corocznych wyzwań placówek oświatowych jest ułożenie dobrego planu lekcji.
    Pomimo postępu technologicznego, proces ten nadal często opiera się na czasochłonnej, ręcznej pracy, co prowadzi do istotnych trudności logistycznych, wydłuża czas przygotowań i może skutkować nieoptymalnymi rozwiązaniami.
    Problem można podzielić na dwa zasadnicze etapy:

    \begin{itemize}
        \item \textbf{Tworzenie przydziału godzinowego} czyli określenie, ile godzin tygodniowo każdy nauczyciel ma prowadzić z daną klasą i w ramach jakiego przedmiotu.
        Jest to zadanie strategiczne, wykonywane na podstawie szacunków liczby uczniów w danym roczniku, które wyznacza szkielet całego planu.
        \item \textbf{Harmonogramowanie} jest drugim i bardziej złożonym etapem. 
        Polega na przypisanie każdej z zaplanowanych lekcji konkretnego dnia, godziny rozpoczęcia i sali, przy ścisłym respektowaniu wszystkich fizycznych, prawnych i jakościowych ograniczeń.
        To właśnie ten etap, realizowany tradycyjnie metodami ręcznymi, jest niezwykle czasochłonny i stanowi największą trudność w organizacji roku szkolnego.
    \end{itemize}

    Niniejsza praca odpowiada na to wyzwanie, prezentując kompleksowe, automatyczne rozwiązanie obejmujące obydwa te etapy.
    Głównym celem pracy jest usprawnienie całego procesu poprzez zaprojektowanie i implementację algorytmu oraz funkcjonalnej aplikacji wspomagającej.

    \section{Cel i zakres pracy}
        Głównym celem pracy jest zaprojektowanie i implementacja kompleksowego systemu wspomagającego automatyczne układanie szkolnego planu lekcji.
        System ten obejmuje funkcjonalną aplikację webową służącą do definiowania wszystkich danych wejściowych i zarządzania procesem, oraz algorytm optymalizacyjny generujący finalne plany.

        Kluczową innowacją jest dekompozycja złożonego problemu harmonogramowania i zastosowanie hybrydowego podejścia, łączącego trzy techniki:
        \begin{itemize}
            \item algorytm zachłanny do tworzenia struktury bloków lekcyjnych,
            \item algorytm ewolucyjny do optymalizacji rozkładu godzinowego na przestrzeni tygodnia oraz
            \item programowania z ograniczeniami do szczegółowego harmonogramowania.
        \end{itemize}
        Takie połączenie ma na celu znalezienie praktycznego kompromisu między wysoką jakością rozwiązania a akceptowalnym czasem obliczeń.

        \newpage
        Aplikacja wraz z algorytmem ma umożliwiać generowanie poprawnych i użytecznych planów lekcji, które spełniają następujące kategorie ograniczeń:
        \begin{itemize}
            \item ograniczenia fizyczne (dostępność sal, unikanie kolizji zajęć),
            \item ograniczenia jakościowe (ciągłość zajęć, brak okienek),
            \item ograniczenia specyficzne definiowane przez użytkownika (bloki lekcyjne).
        \end{itemize}

        Dla osiągnięcia postawionego celu konieczne jest wykonanie następujących zadań:
        \begin{itemize}
            \item zaprojektowanie i implementacja relacyjnej bazy danych do przechowywania wszystkich zasobów (nauczyciele, klasy, sale, przedmioty), ograniczeń, definicji bloków oraz wyników generowania planów,
            \item zaprojektowanie i implementacja interfejsu użytkownika umożliwiającego intuicyjne wprowadzanie danych, konfigurację parametrów algorytmu oraz wizualizację wygenerowanych planów z perspektywy nauczyciela i klasy,
            \item opracowanie i implementacja wieloetapowego algorytmu generowania planu,
            \item integracja modułu algorytmicznego z aplikacją webową i zapewnienie pełnej funkcjonalności poprzez odpowiedn interfejs programowania aplikacji,
            \item przeprowadzenie testów i walidacji systemu na rzeczywistych danych szkolnych w celu potwierdzenia jego poprawności, wydajności i praktycznej użyteczności.
        \end{itemize}

    \section{Aspekt systemowy}
        W pracy zastosowałem podejście systemowe, ponieważ stworzenie aplikacji do generowania planu lekcji wymaga spojrzenia wykraczającego poza sam algorytm optymalizacyjny.
        Takie podejście obejmuje analizę wymagań, ich formalizację, projekt i implementację rozwiązania oraz końcową walidację, co pozwala traktować aplikację jako kompletny system informatyczny, a nie jedynie algorytm.

        Podejście systemowe pomogło mi także z dogłębną analizą problemu oraz jego dalszą dekompozycją.
        Modelowanie części algorytmu jako obiektów wejściowo-wyjściowych znacząco ułatwiło proces inżynierski.
        
        Zróżnicowanie uwarunkowań istotnie wpływa na projekt systemu. 
        Należy uwzględnić zarówno wymagania prawne, fizyczne ograniczenia szkoły (dostępność sal i nauczycieli), jak i kwestie związane z komfortem uczniów, takie jak równomierne rozłożenie obciążeń dydaktycznych.
        Wszystkie te czynniki muszą być odzwierciedlone w modelu danych oraz w module optymalizacyjnym.
        
        Istotny jest również aspekt użytkowy: system musi być praktyczny i intuicyjny w obsłudze.
        Dlatego projekt obejmuje także ergonomię interfejsu oraz sposób prezentacji danych.
        
        Podejście systemowe jest konieczne, aby stworzyć spójne, skalowalne i odpowiadające rzeczywistym potrzebom szkoły narzędzie.

\section{Układ pracy}
    Praca dzieli się na sześć zasadniczych rozdziałów, które kolejno wprowadzają w tematykę, przedstawiają stan wiedzy, prezentują autorskie rozwiązanie, opisują jego praktyczną implementację, weryfikują jej skuteczność oraz podsumowują uzyskane rezultaty i wskazują kierunki dalszego rozwoju.

    Rozdział~\ref{chapter:related} poświęcony jest przeglądowi powiązanych prac i stanowi teoretyczne zaplecze dla dalszych rozważań.
    W jego ramach dokonałem analizy istniejących podejść naukowych i komercyjnych.
    Szczególnie dokładnie omówiłem podejście iteracyjne programowania mieszanego całkowitoliczbowego, które było inspiracją dla architektury aplikacji.
    Następnie przedstawiłem klasyczny problem harmonogramowania zadań typu job-shop oraz pokazałem, jak mój problem jest jego odmianą.
    Na koniec, w celu zrozumienia rzeczywistych wymagań i wyzwań, dokonałem szczegółowego przeglądu wiodącego na rynku polskim systemu komercyjnego.

    Centralną i kluczową częścią pracy jest rozdział~\ref{chapter:algorytm}, który w całości dotyczy problemu układania planu lekcji i proponowanego algorytmu jego rozwiązania.
    Rozdział zaczyna się od formalnego sformułowania problemu optymalizacyjnego, włączając w to definicję terminologii, danych wejściowych, szukanych, ograniczeń i funkcji celu.
    Następnie, w oparciu o doświadczenia z wcześniejszych prób rozwiązania, zaprezentowałem kluczową innowację pracy: hierarchiczną dekompozycję oryginalnego, złożonego problemu na trzy mniejsze etapy.
    Dla każdego z tych etapów opisałem dobór i zasadę działania techniki adekwatnej do danego podproblemu oraz przykładowe wyniki na rzeczywistych danych, co pozwoli na prześledzenie logiki całego procesu generowania planu.

    Rozdział~\ref{chapter:website} koncentruje się na praktycznym aspekcie pracy, czyli projekcie oraz implementacji aplikacji webowej.
    Zaczyna się on od specyfikacji wymagań, przypadków użycia oraz diagramów sekwencji.
    W dalszej części szczegółowo opisałem projekt systemu obejmujący jego ogólną architekturę, projekt bazy danych oraz omówienie projektu interfejsu użytkownika.
    Ostatnia część tego rozdziału opisuje implementację oraz uzasadnia wybór stosu technologicznego.

    Rozdział~\ref{chapter:tests} służy weryfikacji i walidacji stworzonego rozwiązania.
    Jego pierwsza część zawiera formalne scenariusze testowe kluczowych funkcjonalności aplikacji, a następna dokumentuje testy przeprowadzone na ich podstawie.
    Najważniejszą część tego rozdziału stanowi analiza jakościowa wyników.
    Obejmuje ona omówienie ograniczeń posiadanych danych, analizę całościową wygenerowanego planu lekcji, a także ocenę kluczowego wskaźnika --- liczby okienek nauczycieli.
    Dalej szczegółowo analizuję krytyczny przypadek klasy o profilu łączonym oraz omawiam wydajność czasową całego procesu optymalizacyjnego.

    Ostatni rozdział pracy~\ref{chapter:conclusions} stanowi podsumowanie oraz prezentuje perspektywy rozwoju.
    Najpierw omówiłem ograniczenia i problemy bieżącej wersji rozwiązania.
    Następnie zaprezentowałem konkretne kierunki dalszych prac, które pomogą w rozwiązaniu wcześniej wspomnianych problemów.
    Rozdział, a tym samym cała praca, zamknie się sformułowaniem ostatecznych wniosków, potwierdzających trafność przyjętej metody dekompozycji oraz oceniających praktyczną przydatność opracowanego systemu.