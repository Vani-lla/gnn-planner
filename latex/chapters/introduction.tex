\Chapter{Wstęp}\label{chapter:introduction}
    Jednym z corocznych wyzwań placówek oświatowych jest ułożenie dobrego planu lekcji. 
    Jest to nieodłączny proces towarzyszący prowadzeniu szkoły, który do tej pory sprawia problemy nawet najlepszym ośrodkom.
    Można go podzielić na 2 etapy:

    \begin{enumerate}
        \item \textbf{Przydział godzinowy nauczycieli do każdej klasy.} Przydział nauczycieli do klas jest zadaniem, które wykonuje się przed rekrutacją.
            W praktyce oznacza to rozpoczęcie pracy nad planem bez informacji o dokładnej ilości uczniów. 
            Dostępne są jedynie prognozy które pozwalają na określenie ilości klas, co zapewnia to ciągłość nauczania jednego nauczyciela z roku na rok dla danej klasy.
        \item \textbf{Przypisania godzin rozpoczęcia i skończenia lekcji oraz sal, w których będą się odbywać.} Mając już informację o ilości godzin lekcyjnych, które każda klasa musi odbyć z każdym nauczycielem, możemy przejść do tworzenia właściwego planu.
            Głównym ograniczeniem jest stworzony w etapie 1.\ przydział godzin. Istnieje jednak wiele innych ograniczeń, które sprawiają, że proces ręcznego układania takiego planu zajmuje niejednokrotnie dziesiątki godzin.
            W efekcie pierwszy miesiąc w szkołach bywa bardzo chaotyczny ze względu na ciągłe zmiany godzin i sal.
    \end{enumerate}

    Drugi etap jest niczym innym jak harmonogramowaniem z twardymi i miękkimi ograniczeniami.
    Metodą którą będę realizował w tej pracy jest algorytm ewolucyjny, inspirowany biologiczną ewolucją.
    Używa takich samych metod do tworzenia najlepszych osobników jak natura: mutacja, krzyżowanie i selekcja.
    Zastosowanie takiego podejścia pozwala na zaprogramowanie problemu programowania liniowego, który można rozwiązać w satysfakcjonującym czasie.
    Umożliwia to jednoczesne zachowanie wysokiej jakości rozwiązania, poprzez użycie pakietów optymalizacyjnych, oraz zmniejszenie czasu obliczeń w porównaniu do rozwiązań realizowanych tylko i wyłącznie przy użyciu programowania całkowitoliczbowego. 
    
    \section{Cel i zakres pracy}
        Celem pracy jest opracowanie i stworzenie aplikacji webowej wspomagającej układanie planu lekcji,
        która będzie wykorzystywać algorytm ewolucyjny przy automatycznej generacji planu. 
        Aplikacja musi umożliwiać użytkownikowi na automatyczne generowanie planu na podstawie wprowadzonych danych.
        Pod uwagę będą brane ogarniczenia fizyczne takie jak dostępność sal i kolizje zajęć, ograniczenia jakościowe takie jak ciągłośc zajęć, ale także ograniczenia definiowane przez użytkownika.
        Realizując te ograniczenia algorytm powinien minimalizować ilość okienek nauczycieli, tak aby zmniejszyć czas spędzony przez nauczycieli w szkołach.

        \newpage
        \noindent
        Konieczne w tym celu jest:
        \begin{itemize}
            \item Zaprojektowanie i implementacja bazy danych do przechowywania ograniczeń, specyfikacji bloków lekcyjnych oraz końcowych wyników.
            \item Zaprojektowanie i wykonanie interfejsu graficznego, pozwalającego użytkownikowi na wprowadzenie wszystkich potrzebnych ograniczeń oraz zobaczenie końcowego planu lekcji.
            \item Zintegrowanie aplikacji webowej z algorytmem do układania planu lekcji, który będze używał algorytmu genetyczny, algorytmu zachłannego oraz solvera liniowego.
        \end{itemize}

        \textit{Coś o podejściu systemowym?}
        Aplikacja będzie przetestowana na rzeczywistych danych, co pozowli na sprawdzenie rozwiązania i zweryfikowanie, czy algorytm nadaje się do układania tego typu planów.

    
    % \subsection{Dane wejściowe}
    %     \textit{Napisałem tą sekcję jeszcze przed otrzymaniej maila od pana na temat zmiany struktury pracy, także nie wiem jeszcze gdzie ją dam}\\

    %     Aby móc umożliwić użytkownikowi stworzenie ograniczenia głównego (\ref{subsubsection:ograniczenie_glowne}) potrzebujemy określonych danych wejściowych:
    %     \begin{itemize}
    %         \item Lista przedmiotów.
    %         \item Lista nauczycieli oraz przedmioty, które mogą prowadzić.
    %         \item Lista klas.
    %     \end{itemize}

    %     \begin{samepage}
    %         Mając te informacje użytkownik może stworzyć tabelę, która będzie służyć do wprowadzenia tygodniowych godzin lekcyjnych.
    %         Nie jest to wystarczająco, aby ułożyć plan zajęć.
    %         W tym celu potrzebujemy więcej informacji:
    %         \begin{itemize}
    %             \item Informacje o dostępności nauczyciela w dane dni.
    %             \item Informacje o blokach lekcyjnych:
    %             \begin{itemize}
    %                 \item Jakie klasy mogą występować w jednych bloku.
    %                 \item Jakie przedmioty mogą występować w jednym bloku.
    %             \end{itemize}
    %             \item Listę sal oraz przedmiotów, które mogą być w nich prowadzone.
    %         \end{itemize}
    %     \end{samepage}

\section{Układ pracy}
    Zarysuj strukturę swojej pracy dyplomowej. Ogólnie przedstawienie pracy. Przykładowo: ,,Praca dzieli się na $7$ rozdziałów (\dots)''. Rozdział \ref{chapter:politechnika} dotyczy (\dots). Temat został rozwinięty~w~\ref{chapter:podrozdzial}.