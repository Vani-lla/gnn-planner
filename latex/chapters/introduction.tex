\Chapter{Wstęp}\label{chapter:introduction}
    Jednym z corocznych wyzwań placówek oświatowych jest ułożenie dobrego planu lekcji.
    Pomimo postępu technologicznego, proces ten nadal sprawia istotne trudności nawet najlepiej zorganizowanym szkołom. Problem można podzielić na dwa zasadnicze etapy:

    \begin{enumerate}
        \item \textbf{Przydział godzinowy nauczycieli do każdej klasy.} Przydział nauczycieli do klas jest zadaniem, które wykonuje się przed rekrutacją.
            W praktyce oznacza to rozpoczęcie pracy nad planem bez informacji o dokładnej liczbie uczniów. 
            Dostępne są jedynie prognozy które pozwalają na określenie liczby klas, co zapewnia to ciągłość nauczania jednego nauczyciela z roku na rok dla danej klasy.
        \item \textbf{Przypisania godzin rozpoczęcia i zakończenia lekcji oraz sal, w których będą się odbywać.} Mając już informację o liczbie godzin lekcyjnych, które każda klasa musi odbyć z każdym nauczycielem, możemy przejść do tworzenia właściwego planu.
            Głównym ograniczeniem jest stworzony w etapie pierwszym przydział godzin. 
            Proces ten, wykonywany ręcznie, może zajmować dziesiątki godzin pracy, co często skutkuje chaosem organizacyjnym w pierwszych tygodniach roku szkolnego.
    \end{enumerate}

    Drugi etap stanowi klasyczny problem harmonogramowania z ograniczeniami twardymi i miękkimi.
    W niniejszej pracy zaproponowałem podejście łączące algorytm ewolucyjny, inspirowany mechanizmami biologicznej ewolucji (mutacja, krzyżowanie, selekcja), z technikami programowania ograniczeń.
    Takie połączenie umożliwia osiągnięcie wysokiej jakości rozwiązań przy akceptowalnym czasie obliczeń, oferując praktyczny kompromis między optymalnością a wydajnością.
    
    \section{Cel i zakres pracy}
        Głównym celem pracy jest opracowanie i implementacja aplikacji webowej wspomagającej automatyczne układanie planu lekcji z wykorzystaniem algorytmu ewolucyjnego.
        Aplikacja ma umożliwiać generowanie planów uwzględniających:
        \begin{itemize}
            \item ograniczenia fizyczne (dostępność sal, unikanie kolizji zajęć),
            \item ograniczenia jakościowe (ciągłość zajęć, brak okienek),
            \item ograniczenia specyficzne definiowane przez użytkownika (bloki lekcyjne),
        \end{itemize}

        Algorytm ma minimalizować liczbę okienek w planach nauczycieli, redukując tym samym czas ich przebywania w szkole.

        Dla osiągnięcia postawionego celu konieczne jest wykonanie następujących zadań:
        \begin{itemize}
            \item zaprojektowanie i implementacja bazy danych do przechowywania ograniczeń, specyfikacji bloków lekcyjnych oraz wyników generowania planów,
            \item zaprojektowanie i wykonanie interfejsu użytkownika umożliwiającego definiowanie wszystkich niezbędnych ograniczeń oraz wizualizację końcowych planów lekcji,
            \item opracowanie algorytmu do układania planu lekcji wykorzystującego:
            \begin{itemize}
                \item algorytm zachłanny do tworzenia struktury bloków lekcyjnych,
                \item algorytm ewolucyjny do optymalizacji rozkładu godzinowego,
                \item solver dla zadań programowania z ograniczeniami do szczegółowego harmonogramowania,
            \end{itemize}
            \item integracja aplikacji webowej z algorytmem.
        \end{itemize}

    \section{Aspekt systemowy}
        W pracy zastosowałem podejście systemowe, ponieważ stworzenie aplikacji do generowania planu lekcji wymaga spojrzenia wykraczającego poza sam algorytm optymalizacyjny.
        Takie podejście obejmuje analizę wymagań, ich formalizację, projekt i implementację rozwiązania oraz końcową walidację, co pozwala traktować aplikację jako kompletny system informatyczny, a nie jedynie algorytm.

        Podejście systemowe pomogło mi także z dogłębną analizą problemu oraz jego dalszą dekompozycją.
        Modelowanie części algorytmu jako obiektów wejściowo-wyjściowych znacząco ułatwiło proces inżynierski.
        
        Zróżnicowanie uwarunkowań istotnie wpływa na projekt systemu. 
        Należy uwzględnić zarówno wymagania prawne, fizyczne ograniczenia szkoły (dostępność sal i nauczycieli), jak i kwestie związane z komfortem uczniów, takie jak równomierne rozłożenie obciążeń dydaktycznych.
        Wszystkie te czynniki muszą być odzwierciedlone w modelu danych oraz w module optymalizacyjnym.
        
        Istotny jest również aspekt użytkowy: system musi być praktyczny i intuicyjny w obsłudze.
        Dlatego projekt obejmuje także ergonomię interfejsu oraz sposób prezentacji danych.
        
        Podejście systemowe jest konieczne, aby stworzyć spójne, skalowalne i odpowiadające rzeczywistym potrzebom szkoły narzędzie.

\section{Układ pracy}
    Praca dzieli się na sześć zasadniczych rozdziałów, które kolejno wprowadzają w tematykę, przedstawiają stan wiedzy, prezentują autorskie rozwiązanie, opisują jego praktyczną implementację, weryfikują jej skuteczność oraz podsumowują uzyskane rezultaty i wskazują kierunki dalszego rozwoju.

    Rozdział~\ref{chapter:related} poświęcony jest przeglądowi powiązanych prac i stanowi teoretyczne zaplecze dla dalszych rozważań.
    W jego ramach dokonałem analizy istniejących podejść naukowych i komercyjnych.
    Szczególnie dokładnie omówiłem podejście iteracyjne programowania mieszanego całkowitoliczbowego, które było inspiracją dla architektury aplikacji.
    Następnie przedstawiłem klasyczny problem harmonogramowania zadań typu job-shop oraz pokazałem, jak mój problem jest jego odmianą.
    Na koniec, w celu zrozumienia rzeczywistych wymagań i wyzwań, dokonałem szczegółowego przeglądu wiodącego na rynku polskim systemu komercyjnego.

    Centralną i kluczową częścią pracy jest rozdział~\ref{chapter:algorytm}, który w całości dotyczy problemu układania planu lekcji i proponowanego algorytmu jego rozwiązania.
    Rozdział zaczyna się od formalnego sformułowania problemu optymalizacyjnego, włączając w to definicję terminologii, danych wejściowych, szukanych, ograniczeń i funkcji celu.
    Następnie, w oparciu o doświadczenia z wcześniejszych prób rozwiązania, zaprezentowałem kluczową innowację pracy: hierarchiczną dekompozycję oryginalnego, złożonego problemu na trzy mniejsze etapy.
    Dla każdego z tych etapów opisałem dobór i zasadę działania techniki adekwatnej do danego podproblemu oraz przykładowe wyniki na rzeczywistych danych, co pozwoli na prześledzenie logiki całego procesu generowania planu.

    Rozdział~\ref{chapter:website} koncentruje się na praktycznym aspekcie pracy, czyli projekcie oraz implementacji aplikacji webowej.
    Zaczyna się on od specyfikacji wymagań, przypadków użycia oraz diagramów sekwencji.
    W dalszej części szczegółowo opisałem projekt systemu obejmujący jego ogólną architekturę, projekt bazy danych oraz omówienie projektu interfejsu użytkownika.
    Ostatnia część tego rozdziału opisuje implementację oraz uzasadnia wybór stosu technologicznego.

    Rozdział~\ref{chapter:tests} służy weryfikacji i walidacji stworzonego rozwiązania.
    Jego pierwsza część zawiera formalne scenariusze testowe kluczowych funkcjonalności aplikacji, a następna dokumentuje testy przeprowadzone na ich podstawie.
    Najważniejszą część tego rozdziału stanowi analiza jakościowa wyników.
    Obejmuje ona omówienie ograniczeń posiadanych danych, analizę całościową wygenerowanego planu lekcji, a także ocenę kluczowego wskaźnika --- liczby okienek nauczycieli.
    Dalej szczegółowo analizuję krytyczny przypadek klasy o profilu łączonym oraz omawiam wydajność czasową całego procesu optymalizacyjnego.

    Ostatni rozdział pracy~\ref{chapter:conclusions} stanowi podsumowanie oraz prezentuje perspektywy rozwoju.
    Najpierw omówiłem ograniczenia i problemy bieżącej wersji rozwiązania.
    Następnie zaprezentowałem konkretne kierunki dalszych prac, które pomogą w rozwiązaniu wcześniej wspomnianych problemów.
    Rozdział, a tym samym cała praca, zamknie się sformułowaniem ostatecznych wniosków, potwierdzających trafność przyjętej metody dekompozycji oraz oceniających praktyczną przydatność opracowanego systemu.