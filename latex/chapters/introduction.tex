\Chapter{Wstęp}\label{chapter:introduction}
    Jednym z corocznych wyzwań placówek oświatowych jest ułożenie dobrego planu lekcji.
    Pomimo postępu technologicznego, proces ten nadal sprawia istotne trudności nawet najlepiej zorganizowanym szkołom. Problem można podzielić na dwa zasadnicze etapy:

    \begin{enumerate}
        \item \textbf{Przydział godzinowy nauczycieli do każdej klasy.} Przydział nauczycieli do klas jest zadaniem, które wykonuje się przed rekrutacją.
            W praktyce oznacza to rozpoczęcie pracy nad planem bez informacji o dokładnej liczbie uczniów. 
            Dostępne są jedynie prognozy które pozwalają na określenie liczby klas, co zapewnia to ciągłość nauczania jednego nauczyciela z roku na rok dla danej klasy.
        \item \textbf{Przypisania godzin rozpoczęcia i zakończenia lekcji oraz sal, w których będą się odbywać.} Mając już informację o liczbie godzin lekcyjnych, które każda klasa musi odbyć z każdym nauczycielem, możemy przejść do tworzenia właściwego planu.
            Głównym ograniczeniem jest stworzony w etapie pierwszym przydział godzin. 
            Proces ten, wykonywany ręcznie, może zajmować dziesiątki godzin pracy, co często skutkuje chaosem organizacyjnym w pierwszych tygodniach roku szkolnego.
    \end{enumerate}

    Drugi etap stanowi klasyczny problem harmonogramowania z ograniczeniami twardymi i miękkimi.
    W niniejszej pracy zaproponowałem podejście łączące algorytm ewolucyjny, inspirowany mechanizmami biologicznej ewolucji (mutacja, krzyżowanie, selekcja), z technikami programowania ograniczeń.
    Takie połączenie umożliwia osiągnięcie wysokiej jakości rozwiązań przy akceptowalnym czasie obliczeń, oferując praktyczny kompromis między optymalnością a wydajnością.
    
    \section{Cel i zakres pracy}
        Głównym celem pracy jest opracowanie i implementacja aplikacji webowej wspomagającej automatyczne układanie planu lekcji z wykorzystaniem algorytmu ewolucyjnego.
        Aplikacja ma umożliwiać generowanie planów uwzględniających:
        \begin{itemize}
            \item Ograniczenia fizyczne (dostępność sal, unikanie kolizji zajęć)
            \item Ograniczenia jakościowe (ciągłość zajęć, brak okienek)
            \item Ograniczenia specyficzne definiowane przez użytkownika (bloki lekcyjne)
        \end{itemize}

        Algorytm ma minimalizować liczbę okienek w planach nauczycieli, redukując tym samym czas ich przebywania w szkole.

        Dla osiągnięcia postawionego celu konieczne jest wykonanie następujących zadań:
        \begin{itemize}
            \item Zaprojektowanie i implementacja bazy danych do przechowywania ograniczeń, specyfikacji bloków lekcyjnych oraz wyników generowania planów.
            \item Zaprojektowanie i wykonanie interfejsu użytkownika umożliwiającego definiowanie wszystkich niezbędnych ograniczeń oraz wizualizację końcowych planów lekcji.
            \item Opracowanie algorytmu do układania planu lekcji wykorzystującego:
            \begin{itemize}
                \item Algorytm zachłanny do tworzenia struktury bloków lekcyjnych.
                \item Algorytm ewolucyjny do optymalizacji rozkładu godzinowego.
                \item Solver programowania ograniczeń do szczegółowego harmonogramowania.
            \end{itemize}
            \item Integracja aplikacji webowej z algorytmem.
        \end{itemize}

        \textit{Coś o podejściu systemowym?}
        Aplikacja będzie przetestowana na rzeczywistych danych, co pozowli na sprawdzenie rozwiązania i zweryfikowanie, czy algorytm nadaje się do układania tego typu planów.

\section{Układ pracy}
    Zarysuj strukturę swojej pracy dyplomowej. Ogólnie przedstawienie pracy. Przykładowo: ,,Praca dzieli się na $7$ rozdziałów (\dots)''. Rozdział \ref{chapter:politechnika} dotyczy (\dots). Temat został rozwinięty~w~\ref{chapter:podrozdzial}.