\Chapter{Wstęp}\label{chapter:introduction}
    \section{Wprowadzenie}
        Jednym z corocznych wyzwań placówek oświatowych jest ułożenie dobrego planu lekcji. 
        Jest to nierozłączny proces towarzyszący prowadzeniu szkoły, który do tej pory sprawia problemy nawet najlepszym ośrodkom.
        Można go podzielić na 2 etapy:

        \begin{enumerate}
            \item \textbf{Przydział godzinowy nauczycieli do każdej klasy.} Jeszcze przed tym jak wiadoma będzie dokładna ilość uczniów, a dostępne są jedynie prognozy,
                potrzebnym jest stworzenie przydziału nauczycieli do klas. 
                Jest to ważne choćby dlatego, aby zapewnić ciągłość nauczania jednego nauczyciela z roku na rok dla jednej klasy.
            \item \textbf{Przydział lekcji do slotów czasowych i sal.} Mając już informację o ilości godzin lekcyjnych, które każda klasa musi odbyć z każdym nauczycielem, możemy przejść do tworzenia właściwego planu.
                Głównym ograniczeniem jest stworzona w etapie 1.\ rozpiska. Istnieje jednak wiele innych ograniczeń, które sprawiają, że proces ręcznego układania takiego harmonogramu zajmuje niejednokrotnie dziesiątki godzin.
                Przez to pierwszy miesiąc w szkołach bardzo często jest bardzo chaotyczny --- ciągłe zmiany godzin i sal.
        \end{enumerate}

        Drugi etap jest niczym innym jak problemem harmonogramowania z twardymi i miękkimi ograniczeniami.
        Na przestrzeni lat wykształcono wiele podejść do rozwiązywania takich problemów. 
        Jednym z nich jest hybrodowe podejście, opierające się na jednoczesnym zastosowaniu metod sztucznej inteligencji oraz metod programowania całkowitoliczbowego.
        Jedną z takich metod jest algorytm ewolucyjny, inspirowany biologiczną ewolucją.
        Używa takich samych metod do tworzenia najlepszych osobników jak natura: mutacja, krzyżowanie i selekcja.
        Zastosowanie takiego podejścia pozwala na stworzenie problemu liniowego, który można rozwiązać w satysfakcjonującym czasie.
        Umożliwia to jednoczesne zachowanie wysokiej jakości rozwiązania, poprzez użycie solvera liniowego, oraz zmniejszenie czasu obliczeń w porównaniu do rozwiązań realizowanych tylko i wyłącznie przy użyciu programowania całkowitoliczbowego. 
    
    \section{Cel i zakres pracy}
        Opracowanie algorytmu ewolucyjnego. Sformułowanie problemu liniowego. Implementacja rozwiązania w Google OR Tools. Przedstawienie dlaczego ta Googlowska biblioteka jest super do tego celu.
        Faktyczna optymalizacja okienek nauczycieli w odróżnieniu do szukania możliwych rozwiązań.


Nowe podejście do formułowania ograniczeń. Nietypowe dane wejściowe. Zastosowanie metody inteligentnej razem z solverem liniowym.
Problemy rozwiązania. Użyte Technologie.


\section{Sformułowanie problemu}
Opisanie danych wejściowych i oczekiwanych danych wyjściowych.

Odwołanie się do konkretnych danych które mam z liceum.

Trzeba ułożyć plan, co jak się dowiedziałem --- jest całkiem trudne. Trzeba zaprojektować bazę danych. Stworzyć strony do aplikacji webowej. Stworzyć backend. Zintegrować wszystko.

\section{Układ pracy}
Zarysuj strukturę swojej pracy dyplomowej. Ogólnie przedstawienie pracy. Przykładowo: ,,Praca dzieli się na $7$ rozdziałów (\dots)''. Rozdział \ref{chapter:politechnika} dotyczy (\dots). Temat został rozwinięty~w~\ref{chapter:podrozdzial}.