\Chapter{Aplikacja}\label{chapter:website}

\section{Specyfikacja wymagań}
    \subsection{Wymagania funkcjonalne}
        Najważniejszym aspektem działania aplikacji są funkcjonalności, które ona posiada.
        Takie wymagania nazywamy funkcjonalnymi --- definiują one konkretne zachowania i funkcje, które system musi porafić wykonać.

        W mojej aplikacji mogę podzielić takie wymagania na 4 grupy:
        \begin{itemize}
            \item \textbf{Obsługa wielu planów lekcji i konfiguracji:}
            \begin{itemize}
                \item Tworzenie i przechowywanie wielu planów lekcji.
                \item Możliwość pracy z wieloma zestawami zasobów i ograniczeń równolegle.
                Przykładowo dla obecnego roku szkolnego mamy zbiór ograniczeń głównych $\mathcal{W}_{2025/2026}$, ale dla następnego jest to juz zupełnie inne zbiór ograniczeń $\mathcal{W}_{2026/2027}$.
            \end{itemize}
            \item \textbf{Wymagania dotyczące wprowadzania danych:}
            \begin{itemize}
                \item Import plików \verb|txt| oraz \verb|csv| z zasobami i wymaganiami w formacie używanym przez placówkę, która dostarczyła dane na potrzeby tej pracy.
                \item Możliwość ręcznego wprowadzenia i edycji wszystkich typów danych:
                \begin{itemize}
                    \item Nauczyciele, klasy, przedmioty, sale.
                    \item Wymagania główne i bloki przedmiotów.
                    \item Dostępności i możliwości sal.
                \end{itemize}
            \end{itemize}
            \item \textbf{Wymagania dotyczące generowania planu lekcji:}
            \begin{itemize}
                \item Automatyczne generowanie kompletnych planów na podstawie zdefiniowanych ograniczeń.
                \item Możliwość sterowania czasem generowania planu poprzez wprowadzanie liczby generacji w algorytmie ewolucyjnym.
            \end{itemize}
            \item \textbf{Wymagania dotyczące wglądu na utworzone plany:}
            \begin{itemize}
                \item Wymagany jest wgląd na plan z perspektywy:
                \begin{itemize}
                    \item Nauczycieli
                    \item Klas
                \end{itemize}
                \item Wymagana jest możliwość wglądu w wiele różnych planów lekcji.
            \end{itemize}
        \end{itemize}

    \subsection{Wymagania niefunkcjonalne}
        Wymagania niefunkcjonalne określają jakościowe cechy systemu, wpływające na jego użyteczność, wydajność i niezawodność.

        Dla rozpatrywanej aplikacji przyjąłem 2 grupy tych wymagań:

        \begin{itemize}
            \item \textbf{Wymagania wydajnościowe:}
            \begin{itemize}
                \item Czas generowania kompletnego planu nie powinien przekraczać 10 minut dla typowych przypadków.
                \item Aplikacja musi obsługiwać realistyczne rozmiary danych: do 1000 wymagań głównych, 100 nauczycieli, 50 klas i 100 sal.
                \item Zużycie pamięci operacyjnej nie może przekraczać 8GB podczas generowania planu.
            \end{itemize}
            \item \textbf{Wymagania co do interfejsu:}
            \begin{itemize}
                \item Interfejs powinien być intuicyjny dla użytkowników zaznajomionych z arkuszami kalkulacyjnymi.
                \item Spójność interfejsu we wszystkich modułach aplikacji.
                \item Przejrzystość, minimalizm, wygoda i prostota w użytkowaniu interfejsu.
            \end{itemize}
        \end{itemize}

        Uwzględnienie zarówno wymagań funkcjonalnych, jak i niefunkcjonalnych pozwala na stworzenie kompletnej aplikacji, która nie tylko realizuje założone funkcje, ale także zapewnia komfortowe i niezawodne środowisko pracy dla docelowych użytkowników.

    \subsection{Przypadki użycia}

\section{Projekt}
    \subsection{Projekt bazy danych}
    \subsection{Projekt interfejsu}
    \subsection{Sposób integracji z algorytmem}

\section{Implementacja}
    \subsection{Wybór narzędzi}
        Dlaczego React+Django. Dlaczego PostgreSQL
    \subsection{Implementacja bazy danych w Django}
    \subsection{Implementacja API w Django}
    \subsection{Implementacja interfejsu w React}
    \subsection{Integracja z algorytmem}