\section*{Streszczenie}

Problem automatycznego układania szkolnego planu lekcji jest złożonym zagadnieniem optymalizacyjnym, z którym corocznie mierzą się placówki oświatowe.
Konieczność równoczesnego spełnienia wielu ograniczeń --- dostępności nauczycieli, niekolizyjności zajęć, ciągłości lekcji oraz minimalizacji okienek --- sprawia, że ręczne tworzenie planu jest procesem niezwykle czasochłonnym.

Niniejsza praca ma na celu usprawnienie tego procesu poprzez zaprojektowanie i implementację algorytmu oraz funkcjonalnej aplikacji webowej.
Kluczową innowacją jest dekompozycja problemu na trzy sekwencyjne etapy: grupowanie pojedynczych zajęć w bloki lekcyjne, optymalny przydział tych bloków do dni tygodnia za pomocą algorytmu ewolucyjnego oraz finalne przypisanie do nich konkretnych sal i terminów.

Proponowane rozwiązanie zostało w pełni zaimplementowane jako aplikacja webowa używająca Reacta oraz Django.
Sercem systemu jest moduł optymalizacyjny napisany w Pythonie, który do efektywnego rozwiązywania problemu harmonogramowania wykorzystuje bibliotekę Google OR-Tools, a do obliczeń w algorytmie ewolucyjnym bibliotekę NumPy.

\section*{Abstract}

The automatic creation of a school timetable is a complex optimization problem that educational institutions face annually.
The need to simultaneously satisfy multiple constraints --- teacher availability, non-overlapping lessons, continuity of classes, and minimization of teacher gaps --- makes manual schedule creation a laborious process.

This thesis aims to streamline this process by designing and implementing an algorithm and a functional web application.
The key innovation is the decomposition of the problem into three sequential stages: grouping individual lessons into instructional blocks, optimally allocating these blocks to weekdays using an evolutionary algorithm, and the final assignment of specific rooms and time slots to them.

The proposed solution has been fully implemented as a web application using React and Django.
The core of the system is an optimization module written in Python, which utilizes the Google OR-Tools library for efficiently solving the scheduling problem and the NumPy library for computations within the evolutionary algorithm.