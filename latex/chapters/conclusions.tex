\Chapter{Wnioski i perspektywy rozwoju}\label{chapter:conclusions}
    \section{Problemy rozwiązania}
        Choć system realizuje wszystkie cele, które założyłem w tej pracy, jego obecna wersja posiada ograniczenia, które mogą utrudniać wykorzystanie w warunkach rzeczywistych.

        Pierwszym problemem jest obsługa bloków lekcyjnych obejmujących tylko część klasy.
        W praktyce takie zajęcia powinny być umieszczane wyłącznie na początku lub końcu dnia, aby uniknąć powstawania okienek dla uczniów.
        System nie narzuca jeszcze tego warunku, co może prowadzić do niepożądanych okienek.

        Uproszczony jest również mechanizm przypisywania sal.
        Obecnie wykorzystuje on tzw. białe listy obsługiwanych przedmiotów, co jest rozwiązaniem poprawnym, ale mało elastycznym.
        W wielu sytuacjach potrzebne jest rozróżnienie między salą ,,dozwoloną'' a ,,preferowaną''.
        Przykładowo nic nie stoi na przeszkodzie, aby w sytuacji w której brakuje sal zajęcia języka polskiego odbyły się w sali do matematyki; natomiast zajęcia wychowania fizycznego mogą odbyć się tylko i wyłącznie w dedykowanych salach.

        Kolejne ograniczenie dotyczy dostępności nauczycieli, które są definiowane w skali dnia.
        W rzeczywistości dostępności powinny być opisane per slot czasowy.

        Ograniczenia dotyczą również samego procesu generowania planów. 
        Algorytm ewolucyjny stosowany do alokacji bloków lekcyjnych pomiędzy dniami działa poprawnie, jednak jego natura heurystyczna powoduje, że nie gwarantuje optymalności.

        Ponadto generowanie planów lekcji jest realizowane lokalnie (niezależnie dla każdego dnia) co redukuje złożoność obliczeniową, ale może prowadzić do utraty globalnej (tygodniowej) optymalności.
    
    \section{Perspektywy rozwoju}
        Dalszy rozwój aplikacji powinien w pierwszej kolejności skoncentrować się na rozwiązaniu problemów zidentyfikowanych w poprzedniej sekcji.
        Ich eliminacja jest kluczowa, aby system mógł zostać wykorzystany w realnych warunkach szkolnych.
        
        Jednym z naturalnych kierunków rozbudowy jest ponowna analiza sposobu przydzielania lekcji do dni tygodnia.
        Obecne podejście heurystyczne jest szybkie, jednak zastosowanie metod programowania całkowitoliczbowego mogłoby zapewnić rozwiązania bliższe optymalnym, szczególnie przy bardziej wymagających konfiguracjach danych.

        Warto również rozszerzyć model sal lekcyjnych poprzez wprowadzenie funkcji kar i nagród. Pozwoliłoby to nie tylko uwzględniać twarde ograniczenia, ale też modelować preferencje.
        Przykładowo ograniczyć korzystanie z nieoptymalnych pomieszczeń lub promować nowoczesne sale.

        Kolejnym istotnym elementem jest zmiana sposobu reprezentacji dostępności nauczycieli. Zamiast ograniczeń na poziomie dni, powinny być one definiowane w skali slotów czasowych.
        Integracja takich informacji w ostatnim etapie algorytmu umożliwi bardziej realistyczne planowanie.

        Wartym analizy kierunkiem rozwoju jest również dodanie czwartego etapu generowania planu lekcji.
        Mógłby on polegać na zebraniu dziennych harmonogramów i potraktowaniu ich jako punktu wyjścia dla solwera programowania liniowego z ograniczeniami.
        Ponieważ bazowe rozwiązanie jest już poprawne, dodatkowa optymalizacja nie powinna znacząco zwiększyć czasu obliczeń, a może wyraźnie poprawić globalną jakość tygodniowego planu.
        
        Poza stroną algorytmiczną, dalsze prace wymagają rozbudowy warstwy użytkowej.
        System powinien obsługiwać wielu użytkowników, co wymaga implementacji kont opartych o przykładowo adresy e-mail i hasła.
        Pozwoliłoby to na wdrożenie aplikacji na rynku.
        
        Interfejs aplikacji również powinien zostać znacząco rozwinięty.
        Obecna wersja skupia się na funkcjonalności, natomiast pełnoprawne wdrożenie wymaga bardziej intuicyjnego, przejrzystego i przyjaznego użytkownikowi UI.
        W szczególności konieczne jest zapewnienie narzędzi do ręcznej edycji wygenerowanego planu --- poprawy sal, wymiany pojedynczych lekcji czy reorganizacji fragmentów tygodnia.
        Tego typu funkcjonalność jest niezbędna, ponieważ nawet najlepszy algorytm nie jest w stanie uwzględnić wszystkich specyficznych preferencji i wyjątków charakterystycznych dla danej szkoły.

    \section{Wnioski}
        Zastosowany algorytm zachłanny do generowania bloków lekcyjnych okazał się trafnym wyborem.
        Jego prostota znacząco ułatwia konstrukcję ograniczeń i eliminuje potrzebę stosowania bardziej złożonych metod.
        Pozwala również na pojedyncze wywołanie funkcji \verb|NoOverlap| dla danej klasy, ponieważ bloki lekcyjne, w przeciwieństwie do pojedynczych zajęć, nie mogą na siebie zachodzić (przykładowo francuski i niemiecki).
        Dzięki temu proces tworzenia harmonogramu jest bardziej przejrzysty i obliczeniowo efektywny.

        Uzyskany plan lekcji, choć wymagałby pewnych ręcznych korekt, jest wystarczająco bliski rozwiązaniu praktycznemu. 
        Podejście które przedstawiłem, oparte na podziale problemu na trzy mniejsze podproblemy, okazało się efektywne przy dostępnej strukturze danych wejściowych. 
        Koncepcja ta jest na tyle dobrze uzasadniona i elastyczna, że z powodzeniem może być stosowana również w przyszłych działaniach i projektach.

        Praca zawiera propozycję wieloetapowego modelu generowania planu lekcji.
        Pokazałem, że takie połączenie umożliwia uzyskanie poprawnych rozwiązań mimo złożonych zależności między zajęciami.
        Eksperymenty wykazały, że rozdzielenie problemu na trzy podproblemy znacząco poprawia efektywność obliczeń i upraszcza formułowanie ograniczeń.


% Zakończenie, podsumowuje najważniejsze wnioski, podaje możliwości dalszego rozwinięcia wykonanych prac i wskazuje obszar potencjalnego zastosowania pracy.
% Rezultaty pracy mają charakter poznawczy, mogą mieć charakter użytkowy.
% Należy dokonać analizy uzyskanych wyników.
% Rezultaty powinny charakteryzować się oryginalnością, a nawet w pewnym stopniu nowatorstwem.
% Praca zawiera (\dots). Zostało pokazane (\dots). Eksperymenty wykazały (\dots). 
% Tu piszemy wnioski i obserwacje.

% Widzimy, że (\dots). Z tego powodu przyszła praca powinna obejmować (\dots). 