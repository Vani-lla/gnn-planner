\Chapter{Wnioski i perspektywy rozwoju}\label{chapter:conclusions}
Zakończenie, podsumowuje najważniejsze wnioski, podaje możliwości dalszego rozwinięcia wykonanych prac i wskazuje obszar potencjalnego zastosowania pracy. Rezultaty pracy mają charakter poznawczy, mogą mieć charakter użytkowy. Należy dokonać analizy uzyskanych wyników. Rezultaty powinny charakteryzować się oryginalnością, a nawet w pewnym stopniu nowatorstwem. Praca zawiera (\dots). Zostało pokazane (\dots). Eksperymenty wykazały (\dots). Tu piszemy wnioski i obserwacje.

Widzimy, że (\dots). Z tego powodu przyszła praca powinna obejmować (\dots). 

\textbf{Na pewno będę miał sporo rzeczy, które wiem, że będę chciał poprawić w przyszłości.}