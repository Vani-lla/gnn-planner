\Chapter{Weryfikacja i walidacja rozwiązania}\label{chapter:tests}
    Rozdział stanowi krytyczną ocenę wdrożonego systemu, podzieloną na dwie kluczowe perspektywy:
    \begin{enumerate}
        \item Testowanie funkcjonalności --- weryfikacja, czy aplikacja działa zgodnie z założeniami projektowymi. Zawiera formalne scenariusze testowe oraz ich praktyczną realizację.
        \item Analiza jakości wyniku --- walidacja, czy wygenerowane plany lekcji są użyteczne i poprawne jakościowo. Obejmuje porównanie z planem ręcznym, analizę krytycznych przypadków oraz ewaluację kluczowego wskaźnika efektywności, liczby okienek nauczycieli.
    \end{enumerate}

\section{Scenariusze testowe}

    \subsection{Scenariusz testowy generowania planu lekcji z określonymi parametrami}
        Poniższy scenariusz opisuje proces wygenerowania planu lekcji przy użyciu zaimplementowanego modułu optymalizacyjnego.

        \begin{table}[h!]
                \centering
                % \setlength{\tabcolsep}{8pt} % Default value: 6pt
                \renewcommand{\arraystretch}{1.3} % 4efault value: 1
                \begin{tabular}{|p{0.25\textwidth}|p{0.70\textwidth}|}
                    \hline
                    \textbf{Scenariusz} & Generowanie planu lekcji z określonymi parametrami \\ \hline
                    
                    \textbf{Opis} & Test weryfikuje poprawność procesu generowania planu lekcji przy zadanej liczbie generacji oraz pełnym zestawie danych wejściowych. \\ \hline
                    
                    \textbf{Widok początkowy} & Strona główna \\ \hline
                    
                    \multirow{4}{*}{\textbf{Wymagania}}
                    & Wprowadzone zasoby: nauczyciele, klasy, sale, przedmioty \\
                    & Zdefiniowane wymagania główne dla każdej klasy \\
                    & Wprowadzone dostępności nauczycieli \\
                    & Zdefiniowane bloki przedmiotowe \\ \hline
                    
                    \multirow{3}{*}{\textbf{Kroki testowe}} 
                    & 1. Przejdź do widoku generowania planu lekcji \\
                    & 2. Wprowadź liczbę generacji \\
                    & 3. Wybierz dostępny zestaw ograniczeń \\
                    & 4. Uruchom proces generowania planu \\ \hline
                    
                    \textbf{Oczekiwany rezultat} &
                    System generuje kompletny plan lekcji zgodny ze wszystkimi ograniczeniami. System pokazuje użytkownikowi powiadomienie o zakończeniu procesu. \\ \hline
                \end{tabular}
                \caption{Generowanie planu lekcji z określonymi parametrami}
            \end{table}

    \subsection{Scenariusz przeglądu planów lekcji}
        Poniższy scenariusz opisuje proces przeglądania istniejących planów lekcji wygenerowanych w systemie.
            
        \begin{table}[h!]
            \centering
            % \setlength{\tabcolsep}{8pt}
            \renewcommand{\arraystretch}{1.4}
            \begin{tabular}{|p{0.25\textwidth}|p{0.70\textwidth}|}
                \hline
                \textbf{Scenariusz} & Przegląd planów lekcji \\ \hline
                
                \textbf{Opis} & Test weryfikuje możliwość przeglądania wielu wygenerowanych planów lekcji oraz podglądu ich szczegółów z perspektywy klasy i nauczyciela. \\ \hline
                
                \textbf{Widok początkowy} & Strona główna \\ \hline
                
                \multirow{2}{*}{\textbf{Wymagania}}
                & Wprowadzone zasoby: nauczyciele, klasy, sale, przedmioty \\
                & W systemie dostępne są co najmniej dwa wygenerowane plany lekcji \\ \hline
                
                \multirow{4}{*}{\textbf{Kroki testowe}}
                & 1. Przejdź do widoku przeglądania planów lekcji \\
                & 2. Wybierz najnowszy dostępny plan \\
                & 3. Wyświetl plan lekcji dla wybranej klasy \\
                & 4. Wyświetl plan lekcji dla konkretnego nauczyciela \\ \hline
                
                \textbf{Oczekiwany rezultat} &
                Użytkownik może poprawnie przeglądać różne warianty planów lekcji oraz widoki szczegółowe, a system prezentuje dane zgodnie z zapisanym stanem. \\ \hline
            \end{tabular}
            \caption{Scenariusz przeglądu planów lekcji}
        \end{table}
            

    \subsection{Scenariusz importowania i edytowania wymagań głównych}
        Scenariusz opisuje proces masowego importu wymagań głównych oraz ich późniejszej ręcznej edycji.
        
        \begin{table}[h!]
            \centering
            % \setlength{\tabcolsep}{8pt}
            \renewcommand{\arraystretch}{1.4}
            \begin{tabular}{|p{0.25\textwidth}|p{0.70\textwidth}|}
                \hline
                \textbf{Scenariusz} & Importowanie i edytowanie wymagań głównych \\ \hline
                
                \textbf{Opis} & Test weryfikuje poprawność importu danych z pliku CSV oraz poprawność procesu edycji wymagań godzinowych dla klas i nauczycieli. \\ \hline
                
                \textbf{Widok początkowy} & Strona główna \\ \hline
                
                \multirow{2}{*}{\textbf{Wymagania}}
                & Wprowadzone zasoby: nauczyciele, klasy, przedmioty \\
                & Dostępny plik CSV zawierający wymagania główne \\ \hline
                
                \multirow{4}{*}{\textbf{Kroki testowe}}
                & 1. Przejdź do widoku dodawania wymagań głównych \\
                & 2. Przeciągnij plik CSV do aplikacji \\
                & 3. Edytuj wymagania godzinowe dla wybranej klasy i nauczyciela \\
                & 4. Zapisz zmiany \\ \hline
                
                \textbf{Oczekiwany rezultat} &
                System poprawnie importuje dane z pliku CSV oraz umożliwia ich edycję i zapisanie nowych wartości. \\ \hline
            \end{tabular}
            \caption{Scenariusz importowania i edytowania wymagań głównych}
        \end{table}
        

    \subsection{Scenariusz edytowania dostępności nauczycieli}
        Scenariusz opisuje proces modyfikacji dostępności nauczycieli po wcześniejszym zdefiniowaniu wymagań głównych.
        
        \begin{table}[h!]
            \centering
            % \setlength{\tabcolsep}{8pt}
            \renewcommand{\arraystretch}{1.4}
            \begin{tabular}{|p{0.25\textwidth}|p{0.70\textwidth}|}
                \hline
                \textbf{Scenariusz} & Edytowanie dostępności nauczycieli \\ \hline
                
                \textbf{Opis} & Test sprawdza możliwość modyfikacji godzin dostępności nauczycieli z poziomu odpowiedniego widoku systemu. \\ \hline
                
                \textbf{Widok początkowy} & Widok wprowadzania wymagań głównych \\ \hline
                
                \multirow{1}{*}{\textbf{Wymagania}}
                & Wprowadzone zasoby: nauczyciele \\ \hline
                
                \multirow{4}{*}{\textbf{Kroki testowe}}
                & 1. Rozwiń wstążkę nawigacji \\
                & 2. Przejdź do widoku dostępności nauczycieli \\
                & 3. Wybierz dostępny zbiór ograniczeń \\
                & 4. Edytuj dostępność wybranego nauczyciela \\
                & 5. Zapisz zmiany \\ \hline
                
                \textbf{Oczekiwany rezultat} &
                System zapisuje zmodyfikowane dane dostępności. System pokazuje powiadomienie o powodzeniu operacji \\ \hline
            \end{tabular}
            \caption{Scenariusz edytowania dostępności nauczycieli}
        \end{table}

\section{Testowanie aplikacji}
    \subsection{Test generowania planu lekcji z określonymi parametrami}
        \begin{figure}[H]
            \centering
            \includegraphics[width=0.45\textwidth, height=4cm]{images/test1/krok1.png}
            \includegraphics[width=0.45\textwidth, height=4cm]{images/test1/krok2.png}
            \includegraphics[width=0.45\textwidth, height=4cm]{images/test1/krok3.png}
            \includegraphics[width=0.45\textwidth, height=4cm]{images/test1/krok4.png}
            \caption{Zrzuty ekranów wykonanych kroków dla scenariusza generowania planu lekcji}\label{fig:test_1}
        \end{figure}

    \subsection{Test przeglądania planów lekcji}
        \begin{figure}[H]
            \centering
            \includegraphics[width=0.45\textwidth, height=4cm]{images/test2/krok1.png}
            \includegraphics[width=0.45\textwidth, height=4cm]{images/test2/krok2.png}
            \includegraphics[width=0.45\textwidth, height=4cm]{images/test2/krok3.png}
            \includegraphics[width=0.45\textwidth, height=4cm]{images/test2/krok4.png}
            \caption{Zrzuty ekranów wykonanych kroków dla scenariusza generowania planu lekcji}\label{fig:test_2}
        \end{figure}

    \subsection{Test importowania i edytowania wymagań głównych}
        \begin{figure}[H]
            \centering
            \includegraphics[width=0.45\textwidth, height=4cm]{images/test3/krok1.png}
            \includegraphics[width=0.45\textwidth, height=4cm]{images/test3/krok2.png}
            \includegraphics[width=0.45\textwidth, height=4cm]{images/test3/krok3.png}
            \includegraphics[width=0.45\textwidth, height=4cm]{images/test3/krok4.png}
            \caption{Zrzuty ekranów wykonanych kroków dla scenariusza generowania planu lekcji}\label{fig:test_3}
        \end{figure}


\section{Analiza jakości wyników generowania planu lekcji}
    Dane przekazane przez placówkę oświatową pochodzą z początkowej fazy procesu układania planu lekcji i uległy zmianom po uzyskaniu pełnych informacji o uczniach.
    W szczególności nie obejmują one lekcji wychowawczych (LW) oraz zajęć dodanych na końcowym etapie tworzenia planu (np. EO, EZ, \dots).
    W konsekwencji nie było możliwe pełne zamodelowanie tych elementów ani uwzględnienie ich podczas optymalizacji planu.

    Istotnym ograniczeniem była również kwestia ochrony danych osobowych nauczycieli --- nie uzyskałem dostępu do imion i nazwisk, co uniemożliwiło identyfikację sytuacji, w których jeden nauczyciel prowadzi więcej niż jeden przedmiot.
    Tym samym nie można było zamodelować potencjalnych konfliktów wynikających z równoległego prowadzenia zajęć w różnych klasach przez tego samego nauczyciela.

    Uwzględniając powyższe ograniczenia, szczególnej analizie poddałem klasy o profilach łączonych, ponieważ to właśnie dla nich ręczne ułożenie planu jest najbardziej wymagające.
    W moim zbiorze danych klasa IIAC łączy profil psychologiczny z prawnym, co oznacza konieczność równoległego prowadzenia bloków lekcyjnych dotyczących obu ścieżek kształcenia.
    Wygenerowany plan lekcji dla tej klasy przedstawiłem na rysunku~\ref{fig:plan_IIBG}. 
    Dla porównania, na rysunku~\ref{fig:plan_IIBG_koper} zaprezentowałem plan opracowany ręcznie przez szkołę.

    % Dane które dostałem od placówki oświatowej są z początkowej fazy układania planu lekcji.
    % Uległy one zmianom kiedy szkoła dostała dokładne dane o uczniach.
    % Przede wszystkim nie są zapisane lekcje wychowawcze (LW) oraz inne lekcje, które dostały dodane na samym końcu układania planu lekcji (EO, EZ, \dots).
    % Przez to nie mogłem ich zamodelować i zabrać pod uwagę podczas układania planu lekcji.
    % Kolejnym aspektem jest prywatność imień i nazwisk nauczycieli --- nie uzyskałem dostępu do tych chronionych danych.
    % Oznacza to, że w przypadku gdzie jeden nauczyciel uczy 2 różnych przedmiotów, to nie byłem w stanie tego zamodelować.

    % Mając to na uwadze warto przyjrzeć się planom dla klas o łączonych profilach.
    % Są to klasy dla których szczególnie trudno ułożyć ręcznie plan ze względu na niestandardowe bloki lekcyjne.
    % Przykładowo w moich danych klasa IIBG jest klasą, która łączy profil językowy z profilem ekonomicznym.
    % W praktyce oznacza to, że zajęcia z programu jednego profilu muszą odbywać się jednocześnie z zajęciami drugiego profilu.
    % Plan lekcji wygenerowany przez algorytm przedstawiłem na rysunku~\ref{fig:plan_IIBG}, dla porównania plan stworzony ręcznie przez placówkę edukacyjną przedstawiono na rysunku~\ref{fig:plan_IIBG_koper}.

    \begin{figure}[H]
        \centering
        \includegraphics[width=0.9\textwidth]{images/plans/plan_IIBG.png}
        \caption{Wygenerowany plan lekcji dla klasy IIAC}\label{fig:plan_IIBG}
    \end{figure}
    \begin{figure}[H]
        \centering
        \includegraphics[width=0.9\textwidth]{images/plans/plan_IIBG_koper.png}
        \caption{Plan lekcji dla klasy IIAC stworzony przez szkołę}\label{fig:plan_IIBG_koper}
    \end{figure}

    Analizując wygenerowany plan można zauważyć, że algorytm poprawnie odwzorował wymagane bloki lekcyjne, takie jak jednoczesne prowadzenie języka biologii i wiedzy o społeczeństwie, a także prawidłowo przydzielił sale o specjalnym przeznaczeniu (sale gimnastyczne, pracownie informatyczne, \dots).
    Plan charakteryzuje się również równomiernym rozkładem obciążeń dydaktycznych w ciągu tygodnia.

    Kolejnym istotnym aspektem jest prawidłowe zaplanowanie zajęć prowadzonych jednocześnie dla wielu klas.
    Najczęściej występują tu bloki językowe obejmujące język francuski, niemiecki oraz rosyjski.
    Przykład takiego bloku, realizowanego w środy, przedstawiono na rysunku~\ref{fig:plan_blok}.

    % Jak możemy zauważyć wygenerowany plan poprawnie wygenerował niestandardowe bloki lekcyjne, takie jak język niemiecki prowadzony razem z matematyką, oraz poprawnie przypisał odpowiednie sale do przedmiotów (Sale gimnastyczne, sale informatyczne, \dots).
    % Widzimy także równomierny przydział obciążeń dydaktycznych dla każdego dnia.

    % Kolejnym ważnym aspektem są także lekcje, które są prowadzone dla wielu klas.
    % Najczęściej występującymi międzyklasowymi blokami lekcyjnymi są bloki językowe zawierające język francuski, język niemiecki oraz język rosyjski.
    % Przykładowo na rysunku~\ref{fig:plan_blok} pokazany blok rosyjski, francuski, niemiecki odbywający się w środę.

    \begin{figure}[H]
        \centering
        \includegraphics[width=0.2\textwidth]{images/plans/IIIC_blok.png}
        \includegraphics[width=0.2\textwidth]{images/plans/IIID_blok.png}
        \includegraphics[width=0.2\textwidth]{images/plans/IIIF_blok.png}
        \caption{Środowy plan lekcji dla klas IIIC, IIID oraz IIIF}\label{fig:plan_blok}
    \end{figure}

    Aby ocenić jakość wygenerowanych harmonogramów, przeanalizowałem liczbę okienek nauczycieli w planie zoptymalizowanym pod kątem minimalizacji ich liczby oraz w planie, który jedynie spełnia wszystkie ograniczenia, lecz nie został poddany optymalizacji.
    Wyniki zestawiono w tabeli~\ref{tab:statystyki_planow}.
    Ze względu na brak dostępu do szczegółowych danych dotyczących planu ułożonego ręcznie przez szkołę możliwe było jedynie porównanie wariantów wygenerowanych przez algorytm.
    % Należy przeanalizować także liczbę okienek nauczycieli w wygenerowanym planie lekcji.
    % Nie mam dostępu do danych z ręcznie ułozonego planu lekcji, ale mogę porównać liczbę okienek przy minimalizacji funkcji celu do liczby okienek przy zwyczajnych szukaniu rozwiązania, które jest wykonalne a nie optymalne.
    % Najważniejsze statystyki przedstawiono w tabeli~\ref{tab:statystyki_planow}

    \begin{table}[h!]
        \centering
        % \setlength{\tabcolsep}{8pt} % Default value: 6pt
        \renewcommand{\arraystretch}{1.3} % 4efault value: 1
        \begin{tabular}{|p{0.45\textwidth}|c|c|c|c|c|c|}
            \hline
            \textbf{Dzień tygodnia} & Pon & Wt & Śr & Czw & Pt & Całościowo \\ \hline
            \textbf{Liczba bloków} & 132 & 133 & 133 & 135 & 140 & 673 \\ \hline
            \textbf{Liczba okienek przy minimalizacji} & 65 & 41 & 68 & 59 & 72 & 305 \\ \hline
            \textbf{Liczba okienek przy minimalizacji per nauczyciel} & 1.05 & 0.98 & 1.10 & 0.95 & 1.16 & 5.24 \\ \hline
            \textbf{Liczba okienek bez minimalizacji} & 172 & 157 & 141 & 139 & 133 & 742 \\ \hline
            \textbf{Liczba okienek bez minimalizacji per nauczyciel} & 2.77 & 2.53 & 2.27 & 2.24 & 2.15 & 11.96 \\ \hline
        \end{tabular}
        \caption{Statystyki dla generowanych planów}\label{tab:statystyki_planow}
    \end{table}

    Czas potrzebny na wygenerowanie planu zajęć z minimalizacją liczby okienek wyniósł 6 minut i 52 sekundy, co spełnia założenia projektowe.
    Wyniki potwierdzają wysoką skuteczność algorytmu --- średnio około pięciu okienek w ciągu całego tygodnia na nauczyciela można uznać za wynik satysfakcjonujący.