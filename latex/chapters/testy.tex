\Chapter{Testowanie i ewaluacja rozwiązania}\label{chapter:tests}

\section{Scenariusze testowe}

    \subsection{Scenariusz testowy generowania planu lakcji z określonymi parametrami}
        Widok początkowy: strona główna

        Setup:
        \begin{itemize}
            \item Wprowadzone zasoby
            \item Wprowadzone wymagania główne
            \item Wprowadzone dostępności nauczycieli
            \item Wprowadzone bloki przedmiotów
        \end{itemize}

        Kroki:
        \begin{enumerate}
            \item Przejdź do widoku generowania planów lekcji
            \item Wprowadź odpowiednią liczbę generacji
            \item Wygeneruj plan lekji
        \end{enumerate}

    \subsection{Scenariusz przeglądu planów lekcji}
        Widok początkowy: strona główna

        Setup:
        \begin{itemize}
            \item Wprowadzone zasoby
            \item Więcej niż jeden plan lekcji możliwy do przeglądania
        \end{itemize}

        Kroki:
        \begin{enumerate}
            \item Przejdź do widoku przeglądania planów lekcji
            \item Wybierz plan najnowszy plan lekcji
            \item Wybierz widok konkretnej klasy
            \item Wybierz widok konkretnego nauczyciela
        \end{enumerate}

    \subsection{Scenariusz importowania i edytowania wymagań głównych}
        Widok początkowy: strona główna

        Setup:
        \begin{itemize}
            \item 
        \end{itemize}

        Kroki:
        \begin{enumerate}
            \item Przejdź do widoku dodawania wymagań głównych
            \item Przeciągnij plik csv z wymaganiami do aplikacji
            \item Edytuj wymaganie godzinowe danego nauczyciela i klasy
            \item Zapisz wyniki
        \end{enumerate}

    \subsection{Scenariusz edytowania dostępności nauczycieli}
        Widok początkowy: widok wprowadzania wymagań głównych

        Setup:
        \begin{itemize}
            \item Wprowadzone zasoby
            \item Wprowadzone wymagania główne
        \end{itemize}

        Kroki:
        \begin{enumerate}
            \item Rozwiń wstążkę nawigacji
            \item Przejdź do strony wprowadzania dostępności nauczycieli
            \item Edytuj dostępność danego nauczyciela
            \item Zapisz wyniki
        \end{enumerate}

\section{Testowanie aplikacji}
    \subsection{Funkcjonalność 1}
    \subsection{Funkcjonalność 2}
