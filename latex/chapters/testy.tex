\Chapter{Testowanie i ewaluacja rozwiązania}\label{chapter:tests}

\section{Scenariusze testowe}

    \subsection{Scenariusz testowy generowania planu lakcji z określonymi parametrami}
        Poniższy scenariusz opisuje proces wygenerowania planu lekcji przy użyciu zaimplementowanego modułu optymalizacyjnego.

        \begin{table}[h!]
                \centering
                % \setlength{\tabcolsep}{8pt} % Default value: 6pt
                \renewcommand{\arraystretch}{1.3} % 4efault value: 1
                \begin{tabular}{|p{0.25\textwidth}|p{0.70\textwidth}|}
                    \hline
                    \textbf{Scenariusz} & Generowanie planu lekcji z określonymi parametrami \\ \hline
                    
                    \textbf{Opis} & Test weryfikuje poprawność procesu generowania planu lekcji przy zadanej liczbie generacji oraz pełnym zestawie danych wejściowych. \\ \hline
                    
                    \textbf{Widok początkowy} & Strona główna \\ \hline
                    
                    \multirow{4}{*}{\textbf{Wymagania}}
                    & Wprowadzone zasoby: nauczyciele, klasy, sale, przedmioty \\
                    & Zdefiniowane wymagania główne dla każdej klasy \\
                    & Wprowadzone dostępności nauczycieli \\
                    & Zdefiniowane bloki przedmiotowe \\ \hline
                    
                    \multirow{3}{*}{\textbf{Kroki testowe}} 
                    & 1. Przejdź do widoku generowania planu lekcji \\
                    & 2. Wprowadź liczbę generacji \\
                    & 3. Uruchom proces generowania planu \\ \hline
                    
                    \textbf{Oczekiwany rezultat} &
                    System generuje kompletny plan lekcji zgodny z ograniczeniami fizycznymi oraz wymaganiami głównymi. Użytkownik otrzymuje możliwość podglądu i walidacji wygenerowanego planu. \\ \hline
                \end{tabular}
                \caption{Generowanie planu lekcji z określonymi parametrami}
            \end{table}

    \subsection{Scenariusz przeglądu planów lekcji}
        Poniższy scenariusz opisuje proces przeglądania istniejących planów lekcji wygenerowanych w systemie.
            
        \begin{table}[h!]
            \centering
            % \setlength{\tabcolsep}{8pt}
            \renewcommand{\arraystretch}{1.4}
            \begin{tabular}{|p{0.25\textwidth}|p{0.70\textwidth}|}
                \hline
                \textbf{Scenariusz} & Przegląd planów lekcji \\ \hline
                
                \textbf{Opis} & Test weryfikuje możliwość przeglądania wielu wygenerowanych planów lekcji oraz podglądu ich szczegółów z perspektywy klasy i nauczyciela. \\ \hline
                
                \textbf{Widok początkowy} & Strona główna \\ \hline
                
                \multirow{2}{*}{\textbf{Wymagania}}
                & Wprowadzone zasoby: nauczyciele, klasy, sale, przedmioty \\
                & W systemie dostępne są co najmniej dwa wygenerowane plany lekcji \\ \hline
                
                \multirow{4}{*}{\textbf{Kroki testowe}}
                & 1. Przejdź do widoku przeglądania planów lekcji \\
                & 2. Wybierz najnowszy dostępny plan \\
                & 3. Wyświetl plan lekcji dla wybranej klasy \\
                & 4. Wyświetl plan lekcji dla konkretnego nauczyciela \\ \hline
                
                \textbf{Oczekiwany rezultat} &
                Użytkownik może poprawnie przeglądać różne warianty planów lekcji oraz widoki szczegółowe, a system prezentuje dane zgodnie z zapisanym stanem. \\ \hline
            \end{tabular}
            \caption{Scenariusz przeglądu planów lekcji}
        \end{table}
            

    \subsection{Scenariusz importowania i edytowania wymagań głównych}
        Scenariusz opisuje proces masowego importu wymagań głównych oraz ich późniejszej ręcznej edycji.
        
        \begin{table}[h!]
            \centering
            % \setlength{\tabcolsep}{8pt}
            \renewcommand{\arraystretch}{1.4}
            \begin{tabular}{|p{0.25\textwidth}|p{0.70\textwidth}|}
                \hline
                \textbf{Scenariusz} & Importowanie i edytowanie wymagań głównych \\ \hline
                
                \textbf{Opis} & Test weryfikuje poprawność importu danych z pliku CSV oraz poprawność procesu edycji wymagań godzinowych dla klas i nauczycieli. \\ \hline
                
                \textbf{Widok początkowy} & Strona główna \\ \hline
                
                \multirow{2}{*}{\textbf{Wymagania}}
                & Wprowadzone zasoby: nauczyciele, klasy, przedmioty \\
                & Dostępny plik CSV zawierający wymagania główne \\ \hline
                
                \multirow{4}{*}{\textbf{Kroki testowe}}
                & 1. Przejdź do widoku dodawania wymagań głównych \\
                & 2. Przeciągnij plik CSV do aplikacji \\
                & 3. Edytuj wymagania godzinowe dla wybranej klasy i nauczyciela \\
                & 4. Zapisz zmiany \\ \hline
                
                \textbf{Oczekiwany rezultat} &
                System poprawnie importuje dane z pliku CSV oraz umożliwia ich edycję i zapisanie nowych wartości. \\ \hline
            \end{tabular}
            \caption{Scenariusz importowania i edytowania wymagań głównych}
        \end{table}
        

    \subsection{Scenariusz edytowania dostępności nauczycieli}
        Scenariusz opisuje proces modyfikacji dostępności nauczycieli po wcześniejszym zdefiniowaniu wymagań głównych.
        
        \begin{table}[h!]
            \centering
            % \setlength{\tabcolsep}{8pt}
            \renewcommand{\arraystretch}{1.4}
            \begin{tabular}{|p{0.25\textwidth}|p{0.70\textwidth}|}
                \hline
                \textbf{Scenariusz} & Edytowanie dostępności nauczycieli \\ \hline
                
                \textbf{Opis} & Test sprawdza możliwość modyfikacji godzin dostępności nauczycieli z poziomu odpowiedniego widoku systemu. \\ \hline
                
                \textbf{Widok początkowy} & Widok wprowadzania wymagań głównych \\ \hline
                
                \multirow{1}{*}{\textbf{Wymagania}}
                & Wprowadzone zasoby: nauczyciele \\ \hline
                
                \multirow{4}{*}{\textbf{Kroki testowe}}
                & 1. Rozwiń wstążkę nawigacji \\
                & 2. Przejdź do widoku dostępności nauczycieli \\
                & 3. Edytuj dostępność wybranego nauczyciela \\
                & 4. Zapisz zmiany \\ \hline
                
                \textbf{Oczekiwany rezultat} &
                System zapisuje zmodyfikowane dane dostępności, a wprowadzone zmiany są widoczne po ponownym wczytaniu widoku. \\ \hline
            \end{tabular}
            \caption{Scenariusz edytowania dostępności nauczycieli}
        \end{table}

\section{Testowanie aplikacji}
    \subsection{Funkcjonalność 1}
    \subsection{Funkcjonalność 2}


\section{Wyniki i analiza skuteczności}
    \begin{itemize}
        \item Dlaczego moje wyniki są wspaniałe
        \item Średni czas potrzebny na generację planu
    \end{itemize}

    \subsection{Statystyki planu}
        \begin{itemize}
            \item Ilość okienek
            \item Rozkład lekcji w tygodniu
            \item Lekcje początkujące/kończące
            \item Statystyki nauczycieli, godziny w szkole do godzin lekcyjnych (płatnych)
        \end{itemize}

    \subsection{Porównanie z ręcznie ułożonym planem}
        Porównanie z planem, który szkoła ułożyła ręcznie.