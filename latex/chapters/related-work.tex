\Chapter{Powiązane prace}\label{chapter:related}
    Rozdział prezentuje przegląd literatury w trzech wyraźnie oddzielonych obszarach, które stanowią fundament dla rozwiązania zaproponowanego w pracy:
    \begin{itemize}
        \item Podejścia algorytmiczne i metodyczne --- analiza pracy naukowej przedstawiającej zaawansowaną, iteracyjną metodę optymalizacji, która posłużyła jako kluczowa inspiracja architektoniczna.
        \item Podstawy teoretyczne --- formalne wprowadzenie do klasycznego problemu harmonogramowania zadań typu job-shop, który stanowi teoretyczną ramę dla problemu harmonogramowania i uzasadnia użycie technik programowania z ograniczeniami.
        \item Rozwiązania praktyczne i komercyjne --- szczegółowy przegląd wiodącego na rynku polskim systemu, służący zrozumieniu rzeczywistych wymagań, złożoności problemu oraz ustaleniu punktu odniesienia dla oceny praktyczności własnego rozwiązania.
    \end{itemize}

    \section{Inspiracja podejściem iteracyjnym}\label{section:podejscie_iteracyjne}
        Praca ,,A mixed-integer programming approach for solving university course timetabling problems''~\cite{rappos2022mixed} przedstawia innowacyjne podejście do rozwiązywania problemów harmonogramowania poprzez dekompozycję na prostsze podproblemy i iteracyjne dodawanie ograniczeń.
        Autorzy wykorzystują strategię polegającą na:
        \begin{itemize}
            \item Uzyskaniu rozwiązania początkowego z podstawowym zestawem ograniczeń.
            \item Iteracyjnym ,,wstrzykiwaniu'' kolejnych ograniczeń z wykorzystaniem zmiennych sztucznych.
            \item Stosowaniu heurystyk redukcji zmiennych dla kontroli złożoności obliczeniowej.
        \end{itemize}
        
        \subsection{Definicja problemu harmonogramowania}
            W przedstawionym podejściu problem harmonogramowania zajęć uniwersyteckich definiowany jest jako zadanie przydzielenia zajęć do dostępnych sal i terminów oraz przypisania studentów do odpowiednich grup zajęciowych, z uwzględnieniem złożonych ograniczeń i preferencji.
            Głównymi komponentami problemu są:
            
            \begin{itemize}
                \item \textbf{Zajęcia}: Reprezentują pojedyncze lekcje dydaktyczne, które muszą zostać zaplanowane. 
                Każde zajęcie $l$ musi zostać przypisane do slotu czasowego $h$ z listy dopuszczalnych slotów $H_l$ oraz, w wymaganych przypadkach, przypisanie sali $r$ z listy dostępnych pomieszczeń $R_l$.
                \item \textbf{Ograniczenia dystrybucyjne}: Stanowią zbiór reguł określających relacje pomiędzy różnymi zajęciami.
                Wyróżnia się ograniczenia twarde, które muszą być bezwzględnie spełnione, oraz miękkie, których naruszenie generuje kary w funkcji celu.
                Przykładowo wymóg rozpoczynania zajęć w tym samym czasie, zakaz nakładania się terminów czy wymóg zachowania minimalnego odstępu czasowego.
                \item \textbf{Struktura kursów i konfiguracji}: Każdy kurs posiada hierarchiczną strukturę, w ramach której student musi wybrać jedną konfigurację $k$,
                za następnie po jednym zajęciu z każdej części składowej $P_k$ danej konfiguracji.
                \item \textbf{Konflikty studentów}: Powstają gdy student jest przypisany do zajęć odbywających się w tym samym czasie
                lub gdy czas pomiędzy zajęciami jest niewystarczający na przemieszczenie się między salami.
            \end{itemize}
        
        Celem optymalizacji jest znalezienie rozwiązania spełniającego wszystkie twarde ograniczenia przy minimalizacji łącznej kary, uwzględniającej naruszenia ograniczeń miękkich oraz konflikty studentów.

        Kluczowym elementem przedstawionej w pracy metody jest dekompozycja oryginalnego, złożonego problemu na szereg prostszych podproblemów, rozwiązywanych sekwencyjnie w procesie iteracyjnym.
        
            \subsubsection{Podstawowy model MIP}
                Podstawę stanowi sformułowanie problemu mieszanocałkowitoliczbowego (MIP) z czterema typami zmiennych decyzyjnych:
                \begin{itemize}
                    \item zmienne $x_{c,h}$ określają czy zajęcia $l$ są przypisane do terminu $h$,
                    \item zmienne $y_{c,r}$ określają czy zajęcia $l$ są przypisane do sali $r$,
                    \item zmienne $z_{s,l}$ określają czy student $s$ jest przypisany do zajęć $l$,
                    \item zmienne $Z_{s,k}$ określają czy student $s$ wybiera konfigurację $k$.
                \end{itemize}
            
                
            \subsubsection{Iteracyjne wstrzykiwanie ograniczeń}
                Model obejmuje fundamentalne ograniczenia zapewniające poprawność podstawowej struktury rozwiązania, takie jak wymóg przypisania każdych zajęć do dokładnie jednego terminu i sali, spełnienie limitów liczby studentów w grupach oraz zakaz równoczesnego użytkowania tej samej sali przez różne zajęcia.
                Ze względu na jego bardzo dużą złożoność, autorzy stosują podejście wieloetapowe:

                \begin{itemize}
                    \item \textbf{Rozwiązanie początkowe}: Optymalizacja rozpoczyna się od uproszczonego modelu zawierającego jedynie fundamentalne ograniczenia oraz podstawowe składniki funkcji celu.
                    \item \textbf{Dodawanie ograniczeń dystrybucyjnych}: Twarde ograniczenia dystrybucyjne są dodawane do modelu w formie ograniczeń wykluczających niedopuszczalne kombinacje przypisań czasowych $x_{c,h}$ i salowych $y_{c,r}$. 
                    \item \textbf{Obsługa ograniczeń specjalnych}: Dla czterech szczególnie złożonych typów ograniczeń (,,MaxDays'', ,,MaxDayLoad'', ,,MaxBreaks'', ,,MaxBlock'') stosowane jest podejście z użyciem ,,leniwych'' ograniczeń (ang. \textit{lazy constraints}). 
                    Takie ograniczenia są dodawane dopiero gdy zostanie znalezione rozwiązanie całkowitoliczbowe, które je narusza.
                    \item \textbf{Iteracyjne uwzględnianie konfliktów studentów}: Podobne podejście stosowane jest dla uwzględnienia w funkcji celu kar za konflikty studentów.
                    Początkowo pomija się najbardziej złożone składowe związane z konfliktami, a następnie iteracyjnie dodaje się do modelu te ograniczenia, które zostały naruszone w poprzednich uruchomieniach.
                \end{itemize}
            
            \subsubsection{Strategie redukcji złożoności}
                Aby utrzymać złożoność obliczeniową na akceptowalnym poziomie, autorzy stosują zaawansowane strategie redukcji modelu:
                \begin{itemize}
                    \item \textbf{Eliminacja zmiennych}: identyfikacja i usuwanie zmiennych, których wartości można ustalić z góry na podstawie analizy struktury problemu.
                    \item \textbf{Agregacja ograniczeń}: grupowanie podobnych ograniczeń w pojedyncze, bardziej zwarte warunki, co znacząco redukuje rozmiar modelu.
                    \item \textbf{Leniwe ograniczenia}: najbardziej złożone ograniczenia są początkowo pomijane, a następnie dodawane tylko gdy zostały naruszone w poprzednich iteracjach.
                \end{itemize}

                Podejście to pozwala na stopniowe poprawianie jakości rozwiązania przy kontrolowanym wzroście złożoności obliczeniowej, zachowując wykonalność rozwiązania na każdym etapie procesu optymalizacji.

    \section{Problem harmonogramowania zadań typu job-shop}\label{section:jssp}
        Problem harmonogramowania typu job-shop (ang. \textit{Job-Shop Scheduling Problem} --- JSSP) należy do klasycznych zagadnień optymalizacji kombinatorycznej w badaniach operacyjnych i zarządzaniu produkcją.
        Ze względu na szerokie zastosowania inżynierskie był intensywnie analizowany w literaturze. Klasyczny wariant, podsumowany m.in. w ,,A survey of job shop scheduling problem: The types and models''~\cite{xiong2022survey}, można sformalizować następująco.

        Rozważamy zbiór maszyn:
        \[ M = \{ M_1, M_2, \dots, M_m \} \]
        zbiór zadań (prac):
        \[ J = \{ J_1, J_2, \dots, J_n \} \]
        oraz dla każdego zadania $J_i$ liniowo uporządkowaną sekwencję operacji:
        \[ O_i = \{ O_{i1}, O_{i2}, \dots, O_{i n_i} \}, \quad O_{i1} \prec O_{i2} \prec \dots \prec O_{i n_i} \]

        Każda operacja $O_{ij}$ jest przypisana do dokładnie jednej maszyny (oznaczanej $\mu(i,j) \in \{1,\dots,m\}$) oraz posiada znany, dodatni czas przetwarzania $p_{ij} \in \mathbb{N}^+$.

        Zmiennymi decyzyjnymi są czasy rozpoczęcia operacji $S_{ij} \in \mathbb{N}_0$. Definiujemy czas zakończenia operacji jako $C_{ij} = S_{ij} + p_{ij}$, a czas zakończenia zadania $J_i$ jako $C_i = C_{i n_i}$.

        Celem jest minimalizacja maksymalnego czasu zakończenia (ang. \textit{makespan}):
        \[ \min\ C_{\max} \quad \text{przy} \quad C_{\max} \ge C_i \ \forall i \in \{1,\dots,n\} \]

        Model spełnia następujące ograniczenia:
        \begin{itemize}
            \item \textbf{Ograniczenia kolejnościowe (precedencji)}: dla każdego zadania $J_i$ i $j=1,\dots,n_i-1$ musi zachodzić
            \[ S_{i,j+1} \ge S_{ij} + p_{ij}. \]
            \item \textbf{Ograniczenia zasobowe (brak nakładania na tej samej maszynie)}: dla każdej pary różnych operacji $(O_{ij}, O_{kl})$ takich, że $\mu(i,j)=\mu(k,l)$ obowiązuje dysjunkcja czasowa
            \begin{equation}
                S_{ij} + p_{ij} \le S_{kl} \quad \text{lub} \quad S_{kl} + p_{kl} \le S_{ij},
            \end{equation}
            co w praktyce implementuje się poprzez zmienne binarne i duże stałe $M$ (formulacje MIP) albo jako ograniczenia dysjunktywne.
        \end{itemize}

        \subsection{Podobieństwa do problemu układania planu lekcji}
            Problem planowania zajęć szkolnych można postrzegać jako rozszerzenie JSSP z wieloma typami zasobów. Podstawowe podobieństwa strukturalne obejmują:

            \subsubsection{Struktura zasobów}
                W JSSP rozważany jest zbiór maszyn $M$. W problemie planu lekcji występują trzy typy zasobów:
                \[ \{\mathcal{T}, \mathcal{C}, \mathcal{R}\}, \]
                gdzie:
                \begin{itemize}
                    \item $\mathcal{T}$ --- zbiór nauczycieli (zasób typu ,,nauczyciel''),
                    \item $\mathcal{C}$ --- zbiór klas (zasób typu ,,klasa''),
                    \item $\mathcal{R}$ --- zbiór sal (zasób typu ,,sala'').
                \end{itemize}

            \subsubsection{Struktura zadań}
                W klasycznym JSSP każde zadanie $J_i$ składa się z operacji $O_{ij}$ wymagających jednego zasobu (maszyny). W układaniu planu zajęć pojedyncza lekcja wymaga jednoczesnego dostępu do trzech zasobów: nauczyciela, klasy i sali.
                Reprezentację lekcji można zapisać jako
                \[ z_i = (d_i, h_i, c_i, t_i, s_i, r_i), \]
                gdzie $d_i$ to dzień, $h_i$ to slot czasowy (przykładowo zerowy: 7:00--7:45), $c_i$ to klasa, $t_i$ to nauczyciel, $s_i$ to przedmiot, a $r_i$ to sala.

\section{Istniejące kompleksowe rozwiązania}
    Zadanie układania planu lekcji jest powszechnym wyzwaniem dla placówek edukacyjnych na całym świecie, w tym Polsce, co zaowocowało rozwojem komercyjnych rozwiązań tego problemu. 
    Na polskim rynku dominuje kilka systemów, wśród których szczególną pozycję zajmuje firma Vulcan, oferująca zintegrowany system zarządzania oświatą.
    Obok niej funkcjonują takie rozwiązania jak Dobry Plan, czy Librus.

    \subsection{Plan lekcji Optivum}
        Vulcan, jako jeden z najstarszych na rynku dostawców, opracował kompleksowe rozwiązanie obejmujące nie tylko układanie planu lekcji, ale także dziennik elektroniczny, sekretariat i inne moduły zarządzania szkołą.
        Jego popularność w polskich szkołach wynika z lat doświadczenia w dostosowywaniu systemu do specyficznych wymagań polskiego systemu edukacji.

        Aplikacja ,,Plan lekcji Optivum'' firmy Vulcan to zaawansowane narzędzie wspomagające proces tworzenia szkolnego planu zajęć.
        Jego główną zaletą jest elastyczność w definiowaniu skomplikowanych ograniczeń, w tym szczegółowe zarządzanie podziałami uczniów na grupy.

        Proces rozpoczyna się od zaimportowania danych z arkusza organizacyjnego, a kończy na publikacji gotowego planu~\cite{vulcanprzewodnik}.

        Podstawą do tworzenia planu są dane zaimportowane z arkusza organizacyjnego. 
        Kluczowym wymaganiem, jakie zostało mi przedstawione przez liceum, które zaopatrzyło mnie w dane do tej pracy jest możliwość importowania podobnych plików w aplikacji.
        W ten sposób pozbywamy się czasochłonnego procesu ręcznego wprowadzania ograniczeń.

        \begin{figure}[!h]
            \centering
            \includegraphics[width=0.65\textwidth]{images/vulcan/import_arkusz.png}
            \caption{Zrzut ekranu importowania arkusza organizacyjnego do programu ,,Plan lekcji Optivum''~\cite{vulcanprzewodnik}}
        \end{figure}

        W następnym etapie użytkownik definiuje zasoby lokalne zaczynając od sal oraz preferencji.
        Dla zajęć grupowych kluczowe jest poprawne zdefiniowanie sal. 
        Jeśli kilka grup ma korzystać z jednego dużego pomieszczenia (sala gimnastyczna, pracownia), należy je podzielić na części (przykładowo: salagim1, salagim2) i traktować jako odrębne sale.
        Dla zajęć poza szkołą (basen) wykorzystuje się tzw. ,,salę pozorną''.

        \begin{figure}[!h]
            \centering
            \includegraphics[width=0.65\textwidth]{images/vulcan/sale.png}
            \caption{Zrzut ekranu wprowadzania sal do programu ,,Plan lekcji Optivum''~\cite{vulcanprzewodnik}}
        \end{figure}
        
        \begin{figure}[!h]
            \centering
            \includegraphics[width=0.65\textwidth]{images/vulcan/sale_pref2.png}
            \caption{Zrzut ekranu wprowadzania preferencji sal względem przedmiotów do programu ,,Plan lekcji Optivum''~\cite{vulcanprzewodnik}}
        \end{figure}
        
        \begin{figure}[!h]
            \centering
            \includegraphics[width=0.65\textwidth]{images/vulcan/sale_pref1.png}
            \caption{Zrzut ekranu wprowadzania preferencji sal względem nauczycieli do programu ,,Plan lekcji Optivum''~\cite{vulcanprzewodnik}}
        \end{figure}

        \newpage
        Po wprowadzeniu sal oraz ich preferencji użytkownik jest poproszony o wprowadzenie ewentualnych podziałów na bloki.
        Decyzja o rozkładzie godzin w tygodniu ma bezpośredni wpływ na grupy.
        Dla przydziału 4-godzinnego WF-u, podział na bloki 2,1,1 oznacza, że jedna z lekcji (przykładowo dla dziewcząt) będzie dwugodzinnym blokiem, a pozostałe pojedynczymi.
        
        \begin{figure}[!h]
            \centering
            \includegraphics[width=0.65\textwidth]{images/vulcan/bloki.png}
            \caption{Zrzut ekranu wprowadzania bloków przedmiotów do programu ,,Plan lekcji Optivum''~\cite{vulcanprzewodnik}}
        \end{figure}
        
        W kolejnym etapie wprowadzane są terminy odbycia zajęć w poszczególnych klasach.
        Program na bieżąco wylicza ,,Bilans gwiazdek'' --- różnicę między liczbą gwiazdek a minimalną liczbą potrzebną do ułożenia planu.
        Dla oddziałów z podziałami na grupy, prawidłowe rozmieszczenie gwiazdek jest kluczowe, aby uniknąć okienek lub niemożności ułożenia planu.

        \begin{figure}[!h]
            \centering
            \includegraphics[width=0.65\textwidth]{images/vulcan/horyzont.png}
            \caption{Zrzut ekranu wprowadzania terminów zajęć dla poszczególnych klas do programu ,,Plan lekcji Optivum''~\cite{vulcanprzewodnik}}
        \end{figure}
        
        Następnie definiuje się dostępność nauczycieli. 
        Wybrane terminy nauczyciela można zablokować lub wskazać jako szczególnie pożądane poprzez odpowiednio symbole $\ominus$ oraz $\oplus$.

        \begin{figure}[!h]
            \centering
            \includegraphics[width=0.65\textwidth]{images/vulcan/dostepnosc_nauczycieli.png}
            \caption{Zrzut ekranu wprowadzania dostępności nauczycieli do programu ,,Plan lekcji Optivum''~\cite{vulcanprzewodnik}}
        \end{figure}

        Przed automatycznym ułożeniem całego planu, zaleca się ręczne lub automatyczne umieszczenie lekcji uznanych za najtrudniejsze, do których należą zajęcia dzielone na grupy i międzyoddziałowe.

        \begin{figure}[!h]
            \centering
            \includegraphics[width=0.65\textwidth]{images/vulcan/plan_trudne.png}
            \caption{Zrzut ekranu wprowadzania ,,trudnych'' lekcji do programu ,,Plan lekcji Optivum''~\cite{vulcanprzewodnik}}
        \end{figure}

        Po ułożeniu ,,trudnych'' lekcji uruchamia się automat dla całego planu.
        Jeśli automat nie poradzi sobie z ułożeniem wszystkich lekcji (przykładowo z powodu zbyt restrykcyjnych warunków dla grup), należy przeanalizować nieułożone lekcje i złagodzić parametry.
        Czasami pomaga kilkakrotne wykonanie minimalizacji okienek i układania całego planu.

        Narzędzie ,,Plan lekcji Optivum'' jest bardzo obszernym narzędziem oferującym wiele możliwości. 
        Bierze pod uwagę praktycznie każdy możliwy scenariusz, który może wystąpić w polskiej szkole, co jest rezultatem wieloletniej obecności na rynku oraz doświadczenia deweloperów. 

        Ceną uniwersalności, jest konieczność stowrzenia rozbudowanej aplikacji wymagającej od użytkowników definiowania wielu ograniczeń, nawet tych rzadko spotykanych w przeciętnej szkole. 
        Kolejnym kosztem takiego podejścia jest konieczność specjalistycznych szkoleń --- Vulcan oferuje kosztowne szesnastogodzinne szkolenia online poświęcone wyłącznie obsłudze aplikacji do układania planu lekcji.

        Podsumowując, ,,Plan lekcji Optivum'' to doskonałe narzędzie dla dużych placówek, które potrzebują sprawdzonego i kompleksowego rozwiązania.
        Niemniej jednak dla małych i średnich szkół, które nie dysponują odpowiednimi funduszami ani czasem na obsługę tak rozbudowanego systemu, może okazać się zbyt skomplikowane i kosztowne.