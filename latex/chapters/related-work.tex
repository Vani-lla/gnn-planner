\Chapter{Powiązane prace}\label{chapter:related}
    Struktura rozdziału?

    \section{Inspiracja podejściem iteracyjnym}\label{section:podejscie_iteracyjne}
        Praca ,,A mixed-integer programming approach for solving university course timetabling problems''~\cite{rappos2022mixed} przedstawia innowacyjne podejście do rozwiązywania problemów harmonogramowania poprzez dekompozycję na prostsze podproblemy i iteracyjne dodawanie ograniczeń. 
        Autorzy wykorzystują strategię polegającą na:
        \begin{enumerate}
            \item Uzyskaniu rozwiązania początkowego z podstawowym zestawem ograniczeń
            \item Iteracyjnym ,,wstrzykiwaniu'' kolejnych ograniczeń z wykorzystaniem zmiennych sztucznych
            \item Stosowaniu heurystyk redukcji zmiennych dla kontroli złożoności obliczeniowej
            \item Utrzymywaniu wykonalności rozwiązania na każdym etapie
        \end{enumerate}

    \section{Job-Shop Problem}
        Problem harmonogramowania typu \textit{Job-Shop} (JSSP) jest jednym z najbardziej klasycznych i istotnych problemów optymalizacji kombinatorycznej w badaniach operacyjnych i zarządzaniu produkcją.
        Ze względu na bardzo szerokie zastosowania inżynierskie był intensywnie studiowany przez wielu autorów.
        Klasyczny wariant można opisać następująco~\cite{xiong2022survey}.
    
        Posiadamy zbiór maszyn:
        \[ M = \{ M_1, M_2, \dots, M_m \} \]
        oraz zbiór zadań (prac):
        \[ J = \{ J_1, J_2, \dots, J_n \} \]
        Każde zadanie (job) $J_i$ składa się z liniowo uporządkowanej sekwencji operacji:
        \[ O_i = \{ O_{i1}, O_{i2}, \dots, O_{i n_i} \} \]
        które muszą zostać wykonane w ściśle określonej kolejności:
        \[ O_{i1} \prec O_{i2} \prec \dots \prec O_{i n_i} \]
    
        Każda operacja $O_{ij}$ jest przypisana do dokładnie jednej maszyny oraz posiada znany czas przetwarzania:
        \[ p_{ij} \in \mathbb{N}^+ \]
    
        Celem jest wyznaczenie kolejności (uszeregowania) i czasów rozpoczęcia wszystkich operacji na maszynach tak, aby zminimalizować maksymalny czas zakończenia dowolnego zadania (ang. \textit{makespan}):
        \[ C_{\max} = \max_{i} C_i \]
        gdzie
        \[ C_i = S_{i n_i} + p_{i n_i} \]
        a $S_{ij}$ oznacza czas rozpoczęcia operacji $O_{ij}$.
    
        Zmienne decyzyjne:
        \[ S_{ij} \in \mathbb{N}_{0} \quad \forall i=1,\dots,n,\ \forall j=1,\dots,n_i \]
    
        Ograniczenia kolejnościowe:
        \[ S_{i,j+1} \ge S_{ij} + p_{ij} \quad \forall i,j = 1,\dots,n_i-1 \]
    
        Dla każdej pary różnych operacji $(O_{ij}, O_{kl})$ przypisanych do tej samej maszyny zachodzi jedno z dwóch:
        \[ S_{ij} + p_{ij} \le S_{kl} \ \lor\ S_{kl} + p_{kl} \le S_{ij} \]
    
        Funkcja celu:
        \[ F_c = \min C_{\max} \quad \text{przy} \quad C_{\max} \ge S_{i n_i} + p_{i n_i} \ \forall i \]
    
        \subsection{Podobieństwa do problemu układania planu lekcji}
            Problem harmonogramowania dla szkół można rozpatrywać jako specyficzną odmianę JSSP z wieloma zasobami. 
            Podstawowe podobieństwa strukturalne obejmują:
    
            \subsubsection{Struktura zasobów}
                W JSSP mamy zbiór maszyn $M$, natomiast w problemie planu lekcji dysponujemy trzema typami zasobów:
                \[ \left\{\mathcal{T}, \mathcal{C}, \mathcal{R}\right\} \]
                gdzie:
                \begin{itemize}
                    \item $\mathcal{T}$ --- zbiór nauczycieli (maszyny typu ,,nauczyciel'')
                    \item $\mathcal{C}$ --- zbiór klas (maszyny typu ,,klasa)'' 
                    \item $\mathcal{R}$ --- zbiór sal (maszyny typu ,,sala'')
                \end{itemize}
    
            \subsubsection{Struktura zadań}
            W klasycznym JSSP zadania $J_i$ składają się z operacji $O_{ij}$. W problemie planu lekcji:
            \[ \text{Lekcja } z_i = (d_i, h_i, c_i, t_i, s_i, r_i) \]
            można traktować jako operację wymagającą jednoczesnego dostępu do trzech zasobów: nauczyciela, klasy i sali.
            gdzie $d_i$ to dzień lekcji, $h_i$ to godzina rozpoczęcia lekcji, $c_i$ to klasa, $t_i$ to nauczyciel, $s_i$ to przedmiot, a $r_i$ to sala.
    
            \subsubsection{Ograniczenia nakładania się}
            Analogia do ograniczenia w JSSP:
            \[ \forall \text{pary lekcji } (z_i, z_j) \text{ współdzielących zasób } \mathcal{R}: \]
            \[ h_i \neq h_j \ \lor \ d_i \neq d_j \ \lor \ r_i \neq r_j \]
            co odpowiada warunkowi w JSSP:
            \[ S_{ij} + p_{ij} \le S_{kl} \ \lor\ S_{kl} + p_{kl} \le S_{ij} \]

            Analogiczne ograniczenia należałoby definiować dla pozostałych zasobów $\mathcal{T}$ i $\mathcal{C}$.
    
            \subsubsection{Reprezentacja interwałowa}
            Kluczowym podobieństwem jest wykorzystanie zmiennych interwałowych. W JSSP operację $O_{ij}$ reprezentujemy jako interwał:
            \[ I_{ij} = (S_{ij}, p_{ij}, C_{ij}) \]
            gdzie $C_{ij} = S_{ij} + p_{ij}$.
    
            W problemie planu lekcji, lekcję $z_i$ reprezentujemy jako:
            \[ I_i = (h_i, v_i, h_i + v_i) \]
            gdzie $v_i$ to liczba godzin (czas trwania), a $h_i$ to slot rozpoczęcia.
    
        \subsection{Różnice i specyfika problemu szkolnego}
            Pomimo strukturalnych podobieństw, problem układania planu lekcji wprowadza istotne modyfikacje:
    
            \subsubsection{Wielozasobowość}
                Podczas gdy w klasycznym JSSP każda operacja wymaga jednej maszyny, lekcja wymaga \textbf{jednoczesnego} dostępu do trzech zasobów:
                \[ \text{Lekcja } z_i \text{ wymaga: } (t_i, c_i, r_i) \in \mathcal{T} \times \mathcal{C} \times \mathcal{R} \]
    
            \subsubsection{Ograniczenia miękkie i jakościowe}
                Problem szkolny wprowadza bogaty zestaw ograniczeń jakościowych niewystępujących w klasycznym JSSP:
                \begin{itemize}
                    \item Brak okienek dla uczniów (ciągłość zajęć)
                    \item Równomierny rozkład godzin lekcyjnych w tygodniu
                    \item Ograniczenia prawne (max 2 godziny tego samego przedmiotu dziennie)
                    \item Dostępność nauczycieli
                \end{itemize}
    
            Innowacją w podejściu jest wprowadzenie bloków lekcyjnych $\mathcal{L}$, które grupują lekcje $z_i$ odbywające się równocześnie, co stanowi rozszerzenie klasycznej koncepcji operacji w JSSP.
            Umożliwia to znacznie łatwiejsze definiowanie ograniczeń.

        \subsection{Gotowe rozwiązania}
            Istnieje wiele istniejących rozwiązań JSSP, jednak szczególnie inspirujące okazało się podejście zaprezentowane w dokumentacji Google OR-Tools~\cite{googleortoolsjssp}, które wykorzystuje zmienne interwałowe do modelowania operacji.
            
            \begin{figure}[ht]
                \centering
                \includegraphics[width=0.9\textwidth]{images/interval_vars.png}
                \caption{Wizualizacja rozwiązania klasycznego problemu JSSP z wykorzystaniem zmiennych interwałowych~\cite{googleortoolsjssp}}\label{fig:intervals}
            \end{figure}
            Wyraźnie widać te inspiracje w sekcji trzeciej algorytmu dotyczącej solvera CP~(\ref{section:solver}), gdzie zastosowano:
            \begin{itemize}
                \item Reprezentację lekcji jako interwałów czasowych
                \item Ograniczenia NoOverlap dla zasobów (nauczyciele, klasy, sale)
                \item Optional intervals dla przypisań sal
                \item Specjalistyczne ograniczenia ciągłości czasowej
            \end{itemize}

            Kluczowe koncepcje zaadaptowane w mojej pracy:
    
            \subsubsection{Zmienne interwałowe}
                \begin{itemize}
                    \item W JSSP: $\text{IntervalVar}(start, duration, end)$.
                    \item W planie lekcji: $\text{IntervalVar}(h_i, v_i, h_i + v_i)$, gdzie $h_i$ to slot czasowy~(\ref{slownik_oznaczen}), a $v_i$ to długość trwania lekcji.
                \end{itemize}
    
            \subsubsection{Ograniczenie NoOverlap}
                \begin{itemize}
                    \item W JSSP: $\text{AddNoOverlap}([\verb|interval|_1, \dots, \verb|interval|_n])$ dla operacji na tej samej maszynie~\cite{googleortoolsoverlap}.
                    \item W planie lekcji: $\text{AddNoOverlap}$ zastosowane dla:
                    \begin{itemize}
                        \item Wszystkich lekcji tej samej klasy
                        \item Wszystkich lekcji tego samego nauczyciela  
                        \item Wszystkich lekcji w tej samej sali
                    \end{itemize}
                \end{itemize}
    
            \subsubsection{Optional Intervals}
                Rozszerzenie o interwały opcjonalne pozwala na modelowanie przypisań sal do nauczycieli w blokach lekcyjnych:
                \[ \text{OptionalIntervalVar}(start, duration, end, presence) \]
                Takie interwały są brane pod uwagę w ograniczeniach tylko i wyłącznie, jeśli $presence = 1$;
                Oznaczna to, że interwał jest ,,aktywny''.

\section{Istniejące kompleksowe rozwiązania}
    Zadanie układania planu lekcji jest powszechnym wyzwaniem dla placówek edukacyjnych na całym świecie, w tym Polsce, co zaowocowało rozwojem komercyjnych rozwiązań tego problemu. 
    Na polskim rynku dominuje kilka systemów, wśród których szczególną pozycję zajmuje firma Vulcan, oferująca zintegrowany system zarządzania oświatą.
    Obok niej funkcjonują takie rozwiązania jak Dobry Plan, czy Librus.

    \subsection{Plan lekcji Optivum}
        Vulcan, jako jeden z najstarszych na rynku dostawców, opracował kompleksowe rozwiązanie obejmujące nie tylko układanie planu lekcji, ale także dziennik elektroniczny, sekretariat i inne moduły zarządzania szkołą.
        Jego popularność w polskich szkołach wynika z lat doświadczenia w dostosowywaniu systemu do specyficznych wymagań polskiego systemu edukacji.

        Aplikacja ,,Plan lekcji Optivum'' firmy Vulcan to zaawansowane narzędzie wspomagające proces tworzenia szkolnego planu zajęć.
        Jego główną zaletą jest elastyczność w definiowaniu skomplikowanych ograniczeń, w tym szczegółowe zarządzanie podziałami uczniów na grupy.

        Proces rozpoczyna się od zaimportowania danych z arkusza organizacyjnego, a kończy na publikacji gotowego planu~\cite{vulcanprzewodnik}.

        Podstawą do tworzenia planu są dane zaimportowane z arkusza organizacyjnego. 
        Kluczowym wymaganiem, które zostało mi przedstawione przez liceum, które zaopatrzyło mnie w dane do tej pracy jest możliwość importowania podobnych plików w aplikacji.
        W ten sposób pozbywamy się żmudnego procesu ręcznego wprowadzania ograniczeń.

        \begin{figure}[!h]
            \centering
            \includegraphics[width=0.65\textwidth]{images/vulcan/import_arkusz.png}
            \caption{Zrzut ekranu importowania arkusza organizacyjnego do programu ,,Plan lekcji Optivum''~\cite{vulcanprzewodnik}}
        \end{figure}

        W następnym etapie użytkownik definije zasoby lokalne zaczynając od sal oraz preferencji.
        Dla zajęć grupowych kluczowe jest poprawne zdefiniowanie sal. 
        Jeśli kilka grup ma korzystać z jednego dużego pomieszczenia (sala gimnastyczna, pracownia), należy je podzielić na części (przykładowo: salagim1, salagim2) i traktować jako odrębne sale.
        Dla zajęć poza szkołą (basen) wykorzystuje się tzw. ,,salę pozorną''.

        \begin{figure}[!h]
            \centering
            \includegraphics[width=0.65\textwidth]{images/vulcan/sale.png}
            \caption{Zrzut ekranu wprowadzania sal do programu ,,Plan lekcji Optivum''~\cite{vulcanprzewodnik}}
        \end{figure}
        
        \begin{figure}[!h]
            \centering
            \includegraphics[width=0.65\textwidth]{images/vulcan/sale_pref2.png}
            \caption{Zrzut ekranu wprowadzania preferencji sal względem przedmiotów do programu ,,Plan lekcji Optivum''~\cite{vulcanprzewodnik}}
        \end{figure}
        
        \begin{figure}[!h]
            \centering
            \includegraphics[width=0.65\textwidth]{images/vulcan/sale_pref1.png}
            \caption{Zrzut ekranu wprowadzania preferencji sal względem nauczycieli do programu ,,Plan lekcji Optivum''~\cite{vulcanprzewodnik}}
        \end{figure}

        \newpage
        Po wprowadzeniu sal oraz ich preferencji użytkownik jest poproszony o wprowadzenie ewentualnych podziałów na bloki.
        Decyzja o rozkładzie godzin w tygodniu ma bezpośredni wpływ na grupy.
        Dla przydziału 4-godzinnego WF-u, podział na bloki 2,1,1 oznacza, że jedna z lekcji (przykładowo dla dziewcząt) będzie dwugodzinnym blokiem, a pozostałe pojedynczymi.
        
        \begin{figure}[!h]
            \centering
            \includegraphics[width=0.65\textwidth]{images/vulcan/bloki.png}
            \caption{Zrzut ekranu wprowadzania bloków przedmiotów do programu ,,Plan lekcji Optivum''~\cite{vulcanprzewodnik}}
        \end{figure}
        
        W kolejnym etapie wprowadzane są terminy odbycia zajęć w poszczególnych klasach.
        Program na bieżąco wylicza ,,Bilans gwiazdek'' --- różnicę między liczbą gwiazdek a minimalną liczbą potrzebną do ułożenia planu.
        Dla oddziałów z podziałami na grupy, prawidłowe rozmieszczenie gwiazdek jest kluczowe, aby uniknąć okienek lub niemożności ułożenia planu.

        \begin{figure}[!h]
            \centering
            \includegraphics[width=0.65\textwidth]{images/vulcan/horyzont.png}
            \caption{Zrzut ekranu wprowadzania terminów zajęć dla poszczególnych klas do programu ,,Plan lekcji Optivum''~\cite{vulcanprzewodnik}}
        \end{figure}
        
        Następnie definiuje się dostępność nauczycieli. 
        Wybrane terminy nauczyciela można zablokować lub wskazać jako szczególnie pożądane poprzez odpowiednio symbole $\ominus$ oraz $\oplus$.

        \begin{figure}[!h]
            \centering
            \includegraphics[width=0.65\textwidth]{images/vulcan/dostepnosc_nauczycieli.png}
            \caption{Zrzut ekranu wprowadzania dostępności nauczycieli do programu ,,Plan lekcji Optivum''~\cite{vulcanprzewodnik}}
        \end{figure}

        Przed automatycznym ułożeniem całego planu, zaleca się ręczne lub automatyczne umieszczenie lekcji uznanych za najtrudniejsze, do których należą zajęcia dzielone na grupy i międzyoddziałowe.
 
        \begin{figure}[!h]
            \centering
            \includegraphics[width=0.65\textwidth]{images/vulcan/plan_trudne.png}
            \caption{Zrzut ekranu wprowadzania ,,trudnych'' lekcji do programu ,,Plan lekcji Optivum''~\cite{vulcanprzewodnik}}
        \end{figure}

        Po ułożeniu ,,trudnych'' lekcji uruchamia się automat dla całego planu.
        Jeśli automat nie poradzi sobie z ułożeniem wszystkich lekcji (przykładowo z powodu zbyt restrykcyjnych warunków dla grup), należy analizować nieułożone lekcje i łagodzić parametry.
        Czasami pomaga kilkakrotne wykonanie minimalizacji okienek i układania całego planu.

        Narzędzie ,,Plan lekcji Optivum'' jest bardzo obszernym narzędziem oferującym wiele możliwości. 
        Bierze pod uwagę praktycznie każdy możliwy scenariusz, który może wystąpić w polskiej szkole, co jest rezultatem wieloletniej obecności na rynku oraz doświadczenia deweloperów. 

        Aplikacja znakomicie ilustruje cenę uniwersalności, którą jest konieczność stworzenia rozbudowanej aplikacji wymagającej od użytkowników definiowania wielu ograniczeń, nawet tych rzadko spotykanych w przeciętnej szkole. 
        Kolejnym kosztem takiego podejścia jest konieczność specjalistycznych szkoleń --- Vulcan oferuje kosztowne szesnastogodzinne szkolenia online poświęcone wyłącznie obsłudze aplikacji do układania planu lekcji.

        Podsumowując, ,,Plan lekcji Optivum'' to doskonałe narzędzie dla dużych placówek, które potrzebują sprawdzonego i kompleksowego rozwiązania.
        Niemniej jednak dla małych i średnich szkół, które nie dysponują odpowiednimi funduszami ani czasem na obsługę tak rozbudowanego systemu, może okazać się zbyt skomplikowane i kosztowne.