\Chapter{Powiązane prace}\label{chapter:related}
    Na przestrzeni lat wykształcono wiele podejść do rozwiązywania takich problemów. 
    Jednym z nich jest hybrodowe podejście, opierające się na jednoczesnym zastosowaniu metod sztucznej inteligencji oraz metod programowania całkowitoliczbowego.

    \section{A mixed-integer programming approach for solving university course timetabling problems.}
        Rappos, E., Thiémard, E., Robert, S. and Hêche, J.F., 2022.
        A mixed-integer programming approach for solving university course timetabling problems.
        Journal of Scheduling, 25(4), pp.391-404.

        Fajna praca~\cite{rappos2022mixed}, którą luźno się interesowałem --- też rozbijała problem na 2 etapy.
        Też stopniowo rozwiązywała problem i stopniowo dodawała ograniczenia do solvera liniowego.

\section{Istniejące rozwiązania}
    \subsection{Vulcan}
    \subsection{Dobry Plan??}
        Jak będzie trzeba więcej.
    \subsection{Flow-shop scheduling}
        Inspirowałem się rozwiązaniem tego problemu, więc warto zawrzeć.